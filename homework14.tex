\documentclass[fleqn]{article}
\oddsidemargin 0.0in
\textwidth 6.0in
\thispagestyle{empty}
\usepackage{import}
\usepackage{amsmath}
\usepackage{graphicx}
\usepackage{flexisym}
\usepackage{amssymb}
\usepackage{bigints} 
\usepackage[english]{babel}
\usepackage[utf8x]{inputenc}
\usepackage{float}
\usepackage[colorinlistoftodos]{todonotes}

\definecolor{hwColor}{HTML}{AD53BA}

\begin{document}

  \begin{titlepage}

    \newcommand{\HRule}{\rule{\linewidth}{0.5mm}} 

    \center


    \textsc{\LARGE Arizona State University}\\[1.5cm]

    \textsc{\LARGE Mathematical Methods For Physics II }\\[1.5cm] 


    \begin{figure}
      \includegraphics[width=\linewidth]{asu.png}
    \end{figure}


    \HRule \\[0.4cm]
    { \huge \bfseries Homework Fourteen}\\[0.4cm] 
    \HRule \\[1.5cm]

    \textbf{Behnam Amiri}

    \bigbreak

    \textbf{Prof: Cecilia Lunardini}

    \bigbreak


    \textbf{{\large \today}\\[2cm]}

    \vfill

  \end{titlepage}

  \textbf{Part A}
  \begin{enumerate}

    \item Expand the function,
    $$
    f(z) = \frac{1 }{(z - 1) (z - 2)}
    $$
    in powers of $z$ everywhere (that is, for all values of $z$. Different regions in the $z$-plane will call for different expansions; you will have to determine these regions; a graph is recommended). \\
    (Hint: split the function in two terms using partial fractions, and find series for each term. Take inspiration from the examples show in class. )
    
    \item Expand the same function as the exercise above, in powers of $(z - 1)$ everywhere.\\
     (Hint: Think of the function as a product, and note that the factor $1/(z-1)$ is already written as a power of $(z-1)$; therefore you can concentrate on the remaining factor $1/(z-2)$. )
    
    \item Find the Laurent series for each of the following functions about the indicated points and within the region closer to that point than any other singularity is to the indicated point. From the series, determine the order of the function's pole at that point.
    \begin{enumerate}
      \item $$\frac{\cosh z}{z^2}~, \hskip 1truecm z=0$$ . 
      \item $$\frac{e^z}{z^2-1}~, \hskip 1truecm z=1$$.    
    \end{enumerate}
    (Hint: refresh your knowledge of Taylor expansions of exponentials, trigonometric and hyperbolic functions.)
    
  \end{enumerate}

  \pagebreak

  \textbf{Part B}
  \begin{enumerate}

    \item Find the residues of each of the following functions at \emph{all} of their respective poles (in other words, for each function, find the poles and then proceed to calculate the residue at each pole):
    \begin{enumerate}
      \item $$\frac{z}{z^2 + 4z + 3}$$
      
      \item $$\frac{1}{e^z - 1}$$
      
      \item $$\cot z$$
    \end{enumerate}
    (Hints:  For (b), you could try to expand the denominator.) 
    
    
    \item Compute the complex integral,
    $$ \oint_{\mathcal C} \frac{-3z + 4}{z(z-1)(z-2)} dz ,$$
    where ${\mathcal C}$ is the circle of radius $R=3/2$ centered on the origin. \\
    (Hint:  be careful as of what poles are within the given contour). 
    
  \end{enumerate}

\end{document}
