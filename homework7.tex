\documentclass[fleqn]{article}
\oddsidemargin 0.0in
\textwidth 6.0in
\thispagestyle{empty}
\usepackage{import}
\usepackage{amsmath}
\usepackage{graphicx}
\usepackage{flexisym}
\usepackage{amssymb}
\usepackage{bigints} 
\usepackage[english]{babel}
\usepackage[utf8x]{inputenc}
\usepackage{float}
\usepackage[colorinlistoftodos]{todonotes}

\definecolor{hwColor}{HTML}{AD53BA}

\begin{document}

  \begin{titlepage}

    \newcommand{\HRule}{\rule{\linewidth}{0.5mm}}

    \center


    \textsc{\LARGE Arizona State University}\\[1.5cm]

    \textsc{\LARGE Mathematical Methods For Physics II }\\[1.5cm]


    \begin{figure}
      \includegraphics[width=\linewidth]{asu.png}
    \end{figure}


    \HRule \\[0.4cm]
    { \huge \bfseries Homework Seven}\\[0.4cm] 
    \HRule \\[1.5cm]

    \textbf{Behnam Amiri}

    \bigbreak

    \textbf{Prof: Cecilia Lunardini}

    \bigbreak


    \textbf{{\large \today}\\[2cm]}

    \vfill

  \end{titlepage}

  The vertical displacement of the membrane of a circular drum (of radius $a$), with respect to its equilibrium position, is described by the two-dimensional wave equation: 
  $$
  \nabla^2 \psi = \frac{1}{v^2} \frac{\partial ^2 \psi}{\partial t^2}
  $$

  \textcolor{hwColor}{
    The method of separation of variables relies upon the assumption that a function of the form $u(x,t)=\phi(x) G(t)$. 
    will be a solution to a linear homogeneous partial differential equation in $x$ and $t$. This is called a product solution
    and provided the boundary conditions are also linear and homogeneous this will also satisfy the boundary conditions.
    The method of Separation of Variables cannot always be used and even when it can be used 
    it will not always be possible to get much past the first step in the method. \\
  }

  \textcolor{hwColor}{
    This strategy can also be used to solve PDEs in cylindrical polar
    coordinates; $u(\rho, \phi)=R(\rho) \Phi(\phi)$. Then we need to substitute this potential solution into our original PDE, and see
    if we can determine what form $R(\rho)$ and $\Phi(\phi)$ should take.
  }

  \begin{enumerate}
    \item Using the method of the separation of the variables illustrated in class, start solving this equation in (cylindrical) polar coordinates. Your variables will be $\rho$, $\phi$ and $t$, so $\psi=\psi(\rho, 
    \phi, t)$. Obtain  three separate differential equations for the three functions of each variable. 

    \item Discuss any constraints on the sign or values of the separation constants. (Hint: consider that your solution should be single valued for $\phi \rightarrow \phi+ 2\pi$, as explained in class). 

    \item For the functions of $\phi$ and $t$, find the general solution. 

    \item Prove that the equation for the function of $\rho$, $R(\rho)$, can be cast in the form of Bessel's equation.  
  \end{enumerate}

\end{document}
