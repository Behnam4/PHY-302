\documentclass[fleqn]{article}
\oddsidemargin 0.0in
\textwidth 6.0in
\thispagestyle{empty}
\usepackage{import}
\usepackage{amsmath}
\usepackage{graphicx}
\usepackage{flexisym}
\usepackage{amssymb}
\usepackage{bigints} 
\usepackage[english]{babel}
\usepackage[utf8x]{inputenc}
\usepackage{float}
\usepackage[colorinlistoftodos]{todonotes}

\definecolor{hwColor}{HTML}{AD53BA}

\begin{document}

  \begin{titlepage}

    \newcommand{\HRule}{\rule{\linewidth}{0.5mm}}

    \center


    \textsc{\LARGE Arizona State University}\\[1.5cm]

    \textsc{\LARGE Mathematical Methods For Physics II }\\[1.5cm]


    \begin{figure}
      \includegraphics[width=\linewidth]{asu.png}
    \end{figure}


    \HRule \\[0.4cm]
    { \huge \bfseries Homework Seven}\\[0.4cm] 
    \HRule \\[1.5cm]

    \textbf{Behnam Amiri}

    \bigbreak

    \textbf{Prof: Cecilia Lunardini}

    \bigbreak


    \textbf{{\large \today}\\[2cm]}

    \vfill

  \end{titlepage}

  The vertical displacement of the membrane of a circular drum (of radius $a$), with respect to its equilibrium position, is described by the two-dimensional wave equation: 
  $$
  \nabla^2 \psi = \frac{1}{v^2} \frac{\partial ^2 \psi}{\partial t^2}
  $$

  \textcolor{hwColor}{
    The method of separation of variables relies upon the assumption that a function of the form $u(x,t)=\phi(x) G(t)$, 
    will be a solution to a linear homogeneous partial differential equation in $x$ and $t$. This is called a product solution
    and provided the boundary conditions are also linear and homogeneous this will also satisfy the boundary conditions.
    The method of Separation of Variables cannot always be used and even when it can be used 
    it will not always be possible to get much past the first step in the method. \\
  }

  \textcolor{hwColor}{
    This strategy can also be used to solve PDEs in cylindrical polar coordinates;
    $u(\rho, \phi)=R(\rho) \Phi(\phi)$. Then we need to substitute this potential solution into our original PDE, and see
    if we can determine what form $R(\rho)$ and $\Phi(\phi)$ should take.
  }

  \begin{enumerate}
    \item Using the method of the separation of the variables illustrated in class, start solving this equation in (cylindrical) polar coordinates. Your variables will be $\rho$, $\phi$ and $t$, so $\psi=\psi(\rho, 
    \phi, t)$. Obtain  three separate differential equations for the three functions of each variable.

      \textcolor{hwColor}{
        Let's remind ourselves the Laplacian in polar coordinates (3-D). \\
        \\
        $
          \overrightarrow{\nabla}^2 \varPhi=\dfrac{1}{\rho} \dfrac{\partial}{\partial \rho}\left(\rho \dfrac{\partial \varPhi}{\partial \rho}\right)+\dfrac{1}{\rho^2} \dfrac{\partial^2 \varPhi}{\partial \phi^2}+\dfrac{\partial^2 \varPhi}{\partial z^2} \\ \\  
        $
        The vertical displacement of the membrane of a circular drum is described by: \\
        \\
        $
          \dfrac{\partial^2 \psi}{\partial t^2}=v^2  \overrightarrow{\nabla}^2 \psi \\
          \\
          RHS: \\ \\
          v^2\left(\dfrac{1}{\rho}(\dfrac{\partial}{\partial \rho}\rho \dfrac{\partial \psi}{\partial \rho}+\rho \dfrac{\partial^2 \psi}{\partial \rho^2})+\dfrac{1}{\rho^2}\dfrac{\partial^2 \psi}{\partial \phi^2}\right)
          =v^2\left(\dfrac{1}{\rho} \dfrac{\partial \psi}{\partial \rho}+\dfrac{\partial^2 \psi}{\partial \rho^2}+\dfrac{1}{\rho^2} \dfrac{\partial^2 \psi}{\partial \phi^2}\right) \\
          \\
          \\
          \\
          \therefore \dfrac{\partial^2 \psi}{\partial t^2}=v^2\left(\dfrac{1}{\rho} \dfrac{\partial \psi}{\partial \rho}+\dfrac{\partial^2 \psi}{\partial \rho^2}+\dfrac{1}{\rho^2} \dfrac{\partial^2 \psi}{\partial \phi^2}\right)=0 \Longrightarrow -\dfrac{1}{v^2}\psi_{tt}+\left(\rho^{-1} \psi_{\rho}+\psi_{\rho \rho}+\rho^{-2} \psi_{\phi \phi}\right)=0 \\ \\ 
        $
        For $\psi(\rho, \phi, t)$ the assumption is that a function of the form $\mathcal{R}(\rho) \Phi(\phi) \mathcal{T}(t)$ is a solution. Hence, let's just plug them into our partial differential equation. \\
        \\
        $
          -\dfrac{1}{v^2}(\mathcal{R}(\rho) \Phi(\phi) \mathcal{T}(t))_{tt}+\rho^{-1} (\mathcal{R}(\rho) \Phi(\phi) \mathcal{T}(t))_{\rho}+(\mathcal{R}(\rho) \Phi(\phi) \mathcal{T}(t))_{\rho \rho}+\rho^{-2} (\mathcal{R}(\rho) \Phi(\phi) \mathcal{T}(t))_{\phi \phi}=0 \\
          \\
          \mathcal{R}_{\rho \rho} \Phi \mathcal{T}+\mathcal{R}_{\rho} \Phi \mathcal{T}+\mathcal{R} \Phi_{\phi \phi} \mathcal{T} \rho^{-2}-\dfrac{1}{v^2}\mathcal{R} \Phi \mathcal{T}_{tt}=0 \\ \\
        $
        Now by dividing both sides, we can simplify this result. which we will end up with: \\
        \\
        $
          \dfrac{\mathcal{R}_{\rho \rho}}{\mathcal{R}}+\dfrac{\mathcal{R}_{\rho}}{\rho \mathcal{R}}+\dfrac{1}{\rho^2} \dfrac{\Phi_{\phi \phi}}{\Phi}=\dfrac{1}{v^2}\dfrac{\mathcal{T}_{tt}}{\mathcal{T}} \\ \\
        $
        Now we are getting somewhere (hopefully $\ddot\smile$). \\
        \\
        We seek to obtain three separate differential equations for the three functions of each variable. Hence, let's equate each side to a constant. \\ \\ 
        $
          \begin{cases}
            \dfrac{\mathcal{R}_{\rho \rho}}{\mathcal{R}}+\dfrac{\mathcal{R}_{\rho}}{\rho \mathcal{R}}+\dfrac{1}{\rho^2} \dfrac{\Phi_{\phi \phi}}{\Phi}=-\mathcal{A} \\
            \\
            \dfrac{1}{v^2}\dfrac{\mathcal{T}_{tt}}{\mathcal{T}}=-\mathcal{B} 
          \end{cases} \Rightarrow \begin{cases}
            \Phi_{\phi \phi}+\mathcal{A}\Phi=0 \\
            \\
            \mathcal{T}_{tt}+\mathcal{B}\mathcal{T}v^2=0
          \end{cases} \\ \\
        $
        If we plugin the two constants into our PDE, then we get: \\
        \\
        $
          \dfrac{\mathcal{R}_{\rho \rho}}{\mathcal{R}}+\dfrac{\mathcal{R}_{\rho}}{\rho \mathcal{R}}+\dfrac{1}{\rho^2} \dfrac{\Phi_{\phi \phi}}{\Phi}=\dfrac{1}{v^2}\dfrac{\mathcal{T}_{tt}}{\mathcal{T}} \Rightarrow \mathcal{R}_{\rho \rho}+\rho^{-1}\mathcal{R}_{\rho}+\mathcal{R} \mathcal{A}-\rho^{-2} \mathcal{B}\mathcal{R}=0 \\
          \\
          \\
          \begin{cases}
            \Phi_{\phi \phi}+\mathcal{A}\Phi=0 \\
            \\
            \mathcal{T}_{tt}+\mathcal{B}\mathcal{T}v^2=0 \\
            \\
            \mathcal{R}_{\rho \rho}+\rho^{-1}\mathcal{R}_{\rho}+\mathcal{R} \mathcal{A}-\rho^{-2} \mathcal{B}\mathcal{R}=0
          \end{cases} ~ or ~  \begin{cases}
            \Phi^{''}+\mathcal{A}\Phi=0 \\
            \\
            \mathcal{T}^{''}+\mathcal{B}\mathcal{T}v^2=0 \\
            \\
            \mathcal{R}^{''}+\rho^{-1}\mathcal{R}^{'}+\mathcal{R} \mathcal{A}-\rho^{-2} \mathcal{B}\mathcal{R}=0
          \end{cases}
        $
      }

    \item Discuss any constraints on the sign or values of the separation constants. (Hint: consider that your solution should be single valued for $\phi \rightarrow \phi+ 2\pi$, as explained in class).

      \textcolor{hwColor}{
        $~~~$ It is true that the physics of the problem requires that $\lambda$ be negative, but for a more fundamental reason than the technical 
        requirement that the answer involve trigonometeric functions.
        The equation $T^{''}=\lambda T$ says that the acceleration of the string is proportional to the displacement. 
        If the constant $\lambda$ were positive, it would say that the further the string were from the mean position of rest, 
        the faster it would be accelerating, so the string would not be vibrating back and forth, 
        but would rather be zooming away at a faster and faster rate.
        The negative sign in $\lambda$ ensures that the string behave like a simple harmonic oscillator: 
        as it is displaced from its mean position, there is a restoring force that pulls it back towards this mean 
        position (rather than accelerating it further and further away from this position), so that it truly vibrates.It is also because of a negative constant we can use Euler’s formula which is really useful and makes our calculations easier.
        \\
        \\
        I am not sure what to say about what we are given as $\Phi(\phi)=\Phi(\phi+ 2\pi)$. These two are equal (assuming they sines and cosines)
        when $\mathcal{A}$ is a positive integer number.
      }

    \pagebreak

    \item For the functions of $\phi$ and $t$, find the general solution. 

      \textcolor{hwColor}{
        The general solution to a differential equation is the most general form that the 
        solution can take and doesn't take any initial conditions into account. \\
      }

      \textcolor{hwColor}{
        (A): \\ \\
        $\Phi^{''}+\mathcal{A}\Phi=0 ~~,$ Suppose $\Phi=e^k \phi$ then we have $k^2+n^2=0 ~ \Rightarrow k=\pm in$: \\ \\
        $
          \Phi(\phi)=M cos(n \phi)+N sin(n \phi), ~ n\in \mathbb{Z}^{+} 
        $
      }

      \rule{15cm}{1pt}

      \textcolor{hwColor}{
        (B): \\ \\
        $\mathcal{T}^{''}+\mathcal{B}\mathcal{T}v^2=0 ~~,$ Suppose $\mathcal{T}=e^lt$ then we have $l^2+(k v)^2=0 ~ \Rightarrow l=\pm ikv$: \\
        \\
        $
          \mathcal{T}(t)=G cos(kv \phi)+H sin(kv \phi), ~ n\in \mathbb{Z}^{+} 
        $
      }

    \item Prove that the equation for the function of $\rho$, $R(\rho)$, can be cast in the form of Bessel's equation.

      \textcolor{hwColor}{
        Bessel's equation is:\\
        \\
        $
          x^2 \dfrac{d^2 y}{dx^2}+x\dfrac{dy}{dx}+(x^2-v^2)y=0 \\
          \\
          \rule{15cm}{1pt}
          \\
          \\
          \begin{cases}
            \mathcal{R}_{\rho \rho}=\dfrac{d}{d\rho}\dfrac{d \mathcal{R}}{d \rho} \\
            \\
            \gamma =\sqrt{\mathcal{A}} \rho
          \end{cases} \Rightarrow \dfrac{d}{d \gamma}\left(\dfrac{d \mathcal{R}}{d \gamma} \dfrac{d \gamma}{d \rho}\right)\dfrac{d \gamma}{d \rho}=\mathcal{A} \mathcal{R}_{\gamma \gamma} \\
          \\
          \\
          \mathcal{R}^{''}+\rho^{-1}\mathcal{R}^{'}+\mathcal{R} \mathcal{A}-\rho^{-2} \mathcal{B}\mathcal{R}=0
          \\
          \\
          \mathcal{R}_{\rho \rho}+\dfrac{1}{\rho}  \mathcal{R}_{\rho}+\mathcal{R}\left(\mathcal{A}-\dfrac{\mathcal{B}}{\rho^2}\right)=0 \\
          \\
          \\
          \mathcal{A} \mathcal{R}_{\gamma \gamma}+\dfrac{\mathcal{A}}{\gamma} \mathcal{R}_{\gamma}+\mathcal{R}\left(\mathcal{A}-\dfrac{\mathcal{A} n^2}{\gamma^2}\right)=0 \\
          \\
          \\
          \therefore ~ \mathcal{R}_{\gamma \gamma}+\gamma^{-1} \mathcal{R}_{\gamma}+\mathcal{R}(1-\dfrac{n^2}{\gamma^2})=0
        $
      }

  \end{enumerate}

\end{document}
