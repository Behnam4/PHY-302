\documentclass[fleqn]{article}
\oddsidemargin 0.0in
\textwidth 6.0in
\thispagestyle{empty}
\usepackage{import}
\usepackage{amsmath}
\usepackage{graphicx}
\usepackage{flexisym}
\usepackage{amssymb}
\usepackage{bigints} 
\usepackage[english]{babel}
\usepackage[utf8x]{inputenc}
\usepackage{float}
\usepackage[colorinlistoftodos]{todonotes}

\definecolor{hwColor}{HTML}{AD53BA}

\begin{document}

  \begin{titlepage}

    \newcommand{\HRule}{\rule{\linewidth}{0.5mm}} % Defines a new command for the horizontal lines, change thickness here

    \center % Center everything on the page



    \textsc{\LARGE Arizona State University}\\[1.5cm] % Name of your university/college

    \textsc{\LARGE Mathematical Methods For Physics II }\\[1.5cm] % Major heading such as course name


    \begin{figure}
      \includegraphics[width=\linewidth]{asu.png}
    \end{figure}


    \HRule \\[0.4cm]
    { \huge \bfseries Homework Ten}\\[0.4cm] 
    \HRule \\[1.5cm]

    \textbf{Behnam Amiri}

    \bigbreak

    \textbf{Prof: Cecilia Lunardini}

    \bigbreak


    \textbf{{\large \today}\\[2cm]}

    \vfill % Fill the rest of the page with whitespace

  \end{titlepage}

  \textbf{Part A}
  \begin{enumerate}

    \item Expand the generating function of the Legendre Polynomials  to derive the first three Legendre Polynomials. 
    (Hint:  It is probably easiest here to compute the first three terms in the Taylor?s series expansion. One could also use the binomial expansion for  $[1 -( 2hx - h^2 ) ]^{-1/2}$, but keeping track of the terms would be more complicated.)


    \item  Show that the generating function in Eq. 18.15 of the textbook is consistent with Legendre's equation.
    (Hint: Take the appropriate first and second derivatives of both sides of Eq. 18.15 (middle and right expressions), and arrange a combination of the left-hand-side results that will vanish.  You should get to a point where it becomes clear that the $P_l(x)$ that appear in Eq. 18.15 indeed satisfy Legendre equation.)

    \item The Legendre Polynomials enter quite naturally into expressions for the electrical or gravitational potentials of
    point sources through the inverse distance factor,
    \begin{equation}
    \frac{1}{| {\mathbf r}_s -{\mathbf r}_f |}
    \end{equation}

    Let $x$ be the cosine of the angle between ${\mathbf r}_s$ and ${\mathbf r}_f$ , and show that
    \begin{equation}
    \frac{1}{| {\mathbf r}_s -{\mathbf r}_f |} = \frac{1}{r_{>}} \sum^{\infty}_{n=0} \left( \frac{r_{<}}{r_{>}}\right)^n P_n(x)
    \end{equation}
    where $r_{<}$ ($ r_{>}$ ) is the lesser (greater) of $|{\mathbf r}_s|$ and $|{\mathbf r}_f|$.

    (Hint: Write $| {\mathbf r}_s -{\mathbf r}_f |$ as $\sqrt{({\mathbf r}_s -{\mathbf r}_f )\cdot ({\mathbf r}_s -{\mathbf r}_f )}=\sqrt{|{\mathbf r}_s|^2 +|{\mathbf r}_f |^2 -2 {\mathbf r}_s \cdot {\mathbf r}_f }$ and compare with eq. 18.15. Apply convergence requirements in order to assign the  $r_{<}$ and  $ r_{>}$ notation) 

        
    
  \end{enumerate}

  \pagebreak

  \textbf{Part B}
  \begin{enumerate}
    \item Show that, if $P_l(x)$ is the Legendre Polynomial of order $l$, then $P^m_l(x)$  as given in Eq. 18.32 of the textbook satisfies
    the associated Legendre equation. (Hint:  Substitute directly into the associated Legendre equation, and use the product rule of differentiation. Then differentiate Legendre?s equation $m$ times and compare). 


    \item Show, using Eq. 18.33, that
    $Y^{-m}_l (\theta, \phi) = (-1)^m Y^{m\ast }_l (\theta, \phi) $, 
    where $Y^{-m}_l (\theta, \phi)$ is the complex conjugate of $Y^{m\ast }_l (\theta, \phi)$.
  \end{enumerate}

\end{document}
