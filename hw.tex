\documentclass[fleqn]{article}
\oddsidemargin 0.0in
\textwidth 6.0in
\thispagestyle{empty}
\usepackage{import}
\usepackage{amsmath}
\usepackage{graphicx}
\usepackage[english]{babel}
\usepackage[utf8x]{inputenc}
\usepackage{float}
\usepackage[colorinlistoftodos]{todonotes}

\definecolor{hwColor}{HTML}{AD53BA}

\begin{document}

  \begin{titlepage}

    \newcommand{\HRule}{\rule{\linewidth}{0.5mm}} % Defines a new command for the horizontal lines, change thickness here

    \center % Center everything on the page



    \textsc{\LARGE Arizona State University}\\[1.5cm] % Name of your university/college

    \textsc{\LARGE Mathematical Methods For Physics II }\\[1.5cm] % Major heading such as course name


    \begin{figure}
      \includegraphics[width=\linewidth]{asu.png}
    \end{figure}


    \HRule \\[0.4cm]
    { \huge \bfseries Homework One}\\[0.4cm] 
    \HRule \\[1.5cm]

    \textbf{Behnam Amiri}

    \bigbreak

    \textbf{Prof: Cecilia Lunardini}

    \bigbreak


    \textbf{{\large \today}\\[2cm]}

    \vfill % Fill the rest of the page with whitespace

  \end{titlepage}

  \begin{enumerate}
    \item Let $\mathbf{r}(t)$ be the position vector of a particle and $\mathbf{v}(t)$ be its velocity vector, $\mathbf{v}(t) = d\mathbf{r}(t)/dt$. Let $\mathbf{a} = d\mathbf{v}/dt$ be the particle's acceleration. Prove that
      \begin{eqnarray}
      \frac{dr}{dt}=\mathbf{\hat{r}}\cdot \mathbf{v}~,  \\
      \frac{dv}{dt}=\mathbf{\hat{v}}\cdot \mathbf{a}~,  
      \end{eqnarray}
      where $r = |\mathbf{r}|$, $v = |\mathbf{v} |$, $\mathbf{\hat{r}}=\mathbf{r}/r$, and $\mathbf{\hat{v}}=\mathbf{v}/v$.   Describe the motion stipulated by $\mathbf{\hat{r}}\cdot \mathbf{v} = 0$; $\mathbf{\hat{v}}\cdot \mathbf{a}=0$.\\
      (Hint: Start by taking the time derivative of $\mathbf{r}\cdot \mathbf{r}=r^2$ and do steps to obtain eq. (1). Use a similar idea to obtain eq. (2)).

      \textcolor{hwColor}{
        (1): Assume $r=a \hat{i}+b \hat{j}+c \hat{k} $ then $ v=\dfrac{dr}{dt}=\dfrac{da}{dt}\hat{i}+\dfrac{db}{dt}\hat{j}+\dfrac{dc}{dt}\hat{k}$, and $\hat{r}=\dfrac{a \hat{i}+b \hat{j}+c \hat{k}}{\sqrt{a^2+b^2+c^2}}$ \\
        \\
        L.H.S: \\
        $
          \dfrac{dr}{dt}=\dfrac{d(\sqrt{a^2+b^2+c^2})}{dt}=\dfrac{2a\dfrac{da}{dt}+2b\dfrac{db}{dt}+2c\dfrac{dc}{dt}}{2\sqrt{a^2+b^2+c^2}}=\dfrac{a\dfrac{da}{dt}+b\dfrac{db}{dt}+c\dfrac{dc}{dt}}{\sqrt{a^2+b^2+c^2}}
        $
        \\
        \\
        R.H.S: \\
        $
          \hat{r}.v=(\dfrac{a \hat{i}+b \hat{j}+c \hat{k}}{\sqrt{a^2+b^2+c^2}}).(\dfrac{da}{dt}\hat{i}+\dfrac{db}{dt}\hat{j}+\dfrac{dc}{dt}\hat{k})=\dfrac{a}{\sqrt{a^2+b^2+c^2}}\dfrac{da}{dt}+\dfrac{b}{\sqrt{a^2+b^2+c^2}}\dfrac{db}{dt}+\dfrac{c}{\sqrt{a^2+b^2+c^2}}\dfrac{dc}{dt} \\
          \\
          =\dfrac{a\dfrac{da}{dt}+b\dfrac{db}{dt}+c\dfrac{dc}{dt}}{\sqrt{a^2+b^2+c^2}} \\
        $
        \\
        $L.H.S=R.H.S \Rightarrow \dfrac{dr}{dt}=\hat{r}.v$
      }

      \textcolor{hwColor}{
        \rule{15cm}{1pt}
      }

      \textcolor{hwColor}{
        (2): Assume $v=\dfrac{da}{dt} \hat{i}+\dfrac{db}{dt} \hat{j}+\dfrac{dc}{dt} \hat{k}$ then $v=|V|=\sqrt{(\dfrac{da}{dt})^2+(\dfrac{db}{dt})^2+(\dfrac{dc}{dt})^2}$ \\
        \\
        L.H.S: \\
        $
          \dfrac{dv}{dt}=\dfrac{d}{dt}(\sqrt{(\dfrac{da}{dt})^2+(\dfrac{db}{dt})^2+(\dfrac{dc}{dt})^2})=\dfrac{2\dfrac{da}{dt}\dfrac{d^2a}{dt^2}+2\dfrac{db}{dt}\dfrac{d^2b}{dt^2}+2\dfrac{dc}{dt}\dfrac{d^2c}{dt^2}}{2\sqrt{(\dfrac{da}{dt})^2+(\dfrac{db}{dt})^2+(\dfrac{dc}{dt})^2}}=\dfrac{\dfrac{da}{dt}\dfrac{d^2a}{dt^2}+\dfrac{db}{dt}\dfrac{d^2b}{dt^2}+\dfrac{dc}{dt}\dfrac{d^2c}{dt^2}}{\sqrt{(\dfrac{da}{dt})^2+(\dfrac{db}{dt})^2+(\dfrac{dc}{dt})^2}} \\
        $
        \\
        \\
        R.H.S: \\
        $\hat{v}=\dfrac{\dfrac{da}{dt}\hat{i}+\dfrac{db}{dt}\hat{j}+\dfrac{dc}{dt}\hat{k}}{\sqrt{(\dfrac{da}{dt})^2+(\dfrac{db}{dt})^2+(\dfrac{dc}{dt})^2}}$ and $a=\dfrac{d^2a}{dt^2}\hat{i}+\dfrac{d^2b}{dt^2}\hat{j}+\dfrac{d^2c}{dt^2}\hat{k}$ \\
        \\
        \\
        \\
        $
          \hat{v}.a=\dfrac{\dfrac{da}{dt}\dfrac{d^2a}{dt^2}}{\sqrt{(\dfrac{da}{dt})^2+(\dfrac{db}{dt})^2+(\dfrac{dc}{dt})^2}}+\dfrac{\dfrac{db}{dt}\dfrac{d^2b}{dt^2}}{\sqrt{(\dfrac{da}{dt})^2+(\dfrac{db}{dt})^2+(\dfrac{dc}{dt})^2}}+\dfrac{\dfrac{dc}{dt}\dfrac{d^2c}{dt^2}}{\sqrt{(\dfrac{da}{dt})^2+(\dfrac{db}{dt})^2+(\dfrac{dc}{dt})^2}} \\
          \\
          \\
          \hat{v}.a=\dfrac{\dfrac{da}{dt}\dfrac{d^2a}{dt^2}+\dfrac{db}{dt}\dfrac{d^2b}{dt^2}+\dfrac{dc}{dt}\dfrac{d^2c}{dt^2}}{\sqrt{(\dfrac{da}{dt})^2+(\dfrac{db}{dt})^2+(\dfrac{dc}{dt})^2}} \\
          \\
          \\
          L.H.S=R.H.S \Rightarrow \dfrac{dv}{dt}=\hat{v}.a
        $
      }

    \item Prove eq. (10.4) of the textbook.
    
      \textcolor{hwColor}{
        $
          Eq. (10.4) ~~~~ \dfrac{d(\phi a)}{du}=\phi \dfrac{da}{du}+a \dfrac{d\phi}{du}
        $
      }
    
    \item Calculate the length of the curve $\gamma(t) =( t-1, 1-t^2, 2 + \frac{2}{3}t^3)$, $t\in[0,1]$.
    Compare this length with that of the segment of extremes $A=\gamma(0)$ and $B=\gamma(1)$.



    \item Let $\gamma$ be the curve (in two dimensions) defined by $\gamma(t)=(t-\sin t, 
    1-\cos t)$, with $t\in[0,2\pi]$.  Determine the point on $\gamma$ where the curvature $\kappa$ is minimum.


    \item Consider the surface $\Sigma=(x,y, \frac{1}{2}(x^2 + 2 y^2))$. Find the vector $\mathbf{n}(x,y)$ which is orthogonal to $\Sigma$ in a generic point $P$.  Find the particular point on $\Sigma$ for which $\mathbf{n}$ is parallel to the $z$ axis. 

  \end{enumerate}

\end{document}
