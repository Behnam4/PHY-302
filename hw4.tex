\documentclass[fleqn]{article}
\oddsidemargin 0.0in
\textwidth 6.0in
\thispagestyle{empty}
\usepackage{import}
\usepackage{amsmath}
\usepackage{graphicx}
\usepackage[english]{babel}
\usepackage[utf8x]{inputenc}
\usepackage{float}
\usepackage[colorinlistoftodos]{todonotes}

\definecolor{hwColor}{HTML}{AD53BA}

\begin{document}

  \begin{titlepage}

    \newcommand{\HRule}{\rule{\linewidth}{0.5mm}} % Defines a new command for the horizontal lines, change thickness here

    \center % Center everything on the page



    \textsc{\LARGE Arizona State University}\\[1.5cm] % Name of your university/college

    \textsc{\LARGE Mathematical Methods For Physics II }\\[1.5cm] % Major heading such as course name


    \begin{figure}
      \includegraphics[width=\linewidth]{asu.png}
    \end{figure}


    \HRule \\[0.4cm]
    { \huge \bfseries Homework Four}\\[0.4cm] 
    \HRule \\[1.5cm]

    \textbf{Behnam Amiri}

    \bigbreak

    \textbf{Prof: Cecilia Lunardini}

    \bigbreak


    \textbf{{\large \today}\\[2cm]}

    \vfill % Fill the rest of the page with whitespace

  \end{titlepage}

  \begin{enumerate}
    \item Consider the vector field 
      $$
      \mathbf{F}= \left(\frac{zx}{\sqrt{x^2 + y^2} }  , \frac{zy}{\sqrt{x^2 + y^2}}, \frac{1}{z^{3/4}} + \sqrt{x^2 + y^2} \right)~
      $$
      defined for $x\neq 0$, $y\neq 0$ and $z>0$. 
      \begin{enumerate}
        \item Compute the curl of $\mathbf{F}$.
        \item Find a potential field corresponding to $\mathbf{F}$, i.e., a scalar field $\phi$ such that $\mathbf{F}=\mathbf{\nabla}\phi$ in the domain where $\mathbf{F}$ is defined. 
      \end{enumerate}
      (Hint: follow the example shown in class.)


    \item Compute the flux of each of the following vector fields that penetrates a square area
    of side 1 centered on the x-axis at $x=1$. The square's sides are parallel to the y- and z-axes.
      \begin{enumerate}
        \item $\mathbf{V}=xyz\mathbf{i}$.
        \item $\mathbf{V}=y\mathbf{k}+z\mathbf{j}$.
        \item $\mathbf{V}=\mathbf{r}$ (where $\mathbf{r}$ is the position vector )
      \end{enumerate}
      (Hint: Take the unit vector $\mathbf{\hat n}=\mathbf{j} \times \mathbf{k} = \mathbf{i}$ as the positive or ``outward" sense of the surface. Note that, for this exercise, $dS=dy dz$.)


    \item Take $S$ to be the (closed) surface of a sphere of radius r centered at the origin. Compute the flux penetrating it radially outward for each of the following vector fields:
      \begin{enumerate}
        \item $\mathbf{V}=\mathbf{r}$ ;
        \item $\mathbf{V}=2 \mathbf{r}/r^3$ (where $r=|\mathbf{r}|$); 
        \item $\mathbf{V}=yz\mathbf{i}+zx\mathbf{j}+xy\mathbf{k}$.
      \end{enumerate}
      (Hint: What is the simplest way of expressing the orthogonal unit vector $\mathbf{\hat n}$ for this surface? In part (c) you will want to use Spherical Polar coordinates, but do not waste time rewriting $\mathbf{V}$. Make judicious use of scalar products.)

  \end{enumerate}

\end{document}
