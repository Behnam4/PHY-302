\documentclass[fleqn]{article}
\oddsidemargin 0.0in
\textwidth 6.0in
\thispagestyle{empty}
\usepackage{import}
\usepackage{amsmath}
\usepackage{graphicx}
\usepackage{flexisym}
\usepackage{amssymb}
\usepackage{bigints} 
\usepackage[english]{babel}
\usepackage[utf8x]{inputenc}
\usepackage{float}
\usepackage[colorinlistoftodos]{todonotes}

\definecolor{hwColor}{HTML}{AD53BA}

\begin{document}

  \begin{titlepage}

    \newcommand{\HRule}{\rule{\linewidth}{0.5mm}} % Defines a new command for the horizontal lines, change thickness here

    \center % Center everything on the page



    \textsc{\LARGE Arizona State University}\\[1.5cm] % Name of your university/college

    \textsc{\LARGE Mathematical Methods For Physics II }\\[1.5cm] % Major heading such as course name


    \begin{figure}
      \includegraphics[width=\linewidth]{asu.png}
    \end{figure}


    \HRule \\[0.4cm]
    { \huge \bfseries Homework Four}\\[0.4cm] 
    \HRule \\[1.5cm]

    \textbf{Behnam Amiri}

    \bigbreak

    \textbf{Prof: Cecilia Lunardini}

    \bigbreak


    \textbf{{\large \today}\\[2cm]}

    \vfill % Fill the rest of the page with whitespace

  \end{titlepage}

  \textbf{Part A}
  \begin{enumerate}
    \item Consider the vector field 
      $$
      \mathbf{F}= \left(\frac{zx}{\sqrt{x^2 + y^2} }  , \frac{zy}{\sqrt{x^2 + y^2}}, \frac{1}{z^{3/4}} + \sqrt{x^2 + y^2} \right)~
      $$
      defined for $x\neq 0$, $y\neq 0$ and $z>0$. 
      \begin{enumerate}
        \item Compute the curl of $\mathbf{F}$.

          \textcolor{hwColor}{
            The curl of a vector field measures the rotation or spinning effect. \\
            \\
            $
              curl ~ F(x,y,z)=\overrightarrow{\nabla} ~ \times ~ F(x,y,z)=\begin{vmatrix}
                \hat{i} & \hat{j} & \hat{k} \\
                \\
                \dfrac{\partial}{\partial x} & \dfrac{\partial}{\partial y} & \dfrac{\partial}{\partial z} \\
                \\
                \dfrac{zx}{\sqrt{x^2+y^2}} & \dfrac{zy}{\sqrt{x^2+y^2}} & \dfrac{1}{z^{3/4}}+\sqrt{x^2+y^2}
              \end{vmatrix}
              \\
              \\
              \\
              =\hat{i}\left[\dfrac{\partial}{\partial y}(\dfrac{1}{z^{3/4}}+\sqrt{x^2+y^2})-\dfrac{\partial}{\partial z}(\dfrac{zy}{\sqrt{x^2+y^2}})\right] \\ \\   
              -\hat{j}\left[\dfrac{\partial}{\partial x}(\dfrac{1}{z^{3/4}}+\sqrt{x^2+y^2})-\dfrac{\partial}{\partial z}(\dfrac{zx}{\sqrt{x^2+y^2}})\right] \\ \\
              +\hat{k}\left[\dfrac{\partial}{\partial x}(\dfrac{zy}{\sqrt{x^2+y^2}})-\dfrac{\partial}{\partial y}(\dfrac{zx}{\sqrt{x^2+y^2}})\right] \\ \\
              \\
              \\
              =\hat{i}\left[\dfrac{y}{\sqrt{x^2+y^2}}-\dfrac{y}{\sqrt{x^2+y^2}}\right] \\ \\
              -\hat{j}\left[\dfrac{x}{\sqrt{x^2+y^2}}-\dfrac{x}{\sqrt{x^2+y^2}}\right] \\ \\
              +\hat{k}\left[-\dfrac{xyz}{(x^2+y^2)^{3/2}}+\dfrac{xyz}{(x^2+y^2)^{3/2}}\right] \\ \\
              \\
              \overrightarrow{\nabla} ~ \times ~ F(x,y,z)=0 \\ \\
            $
            Since $\mathbf{curl ~ F=0}$, then we can say $\mathbf{F}$ is conservative.
          } 

        \item Find a potential field corresponding to $\mathbf{F}$, i.e., a scalar field $\phi$ such that $\mathbf{F}=\mathbf{\nabla}\phi$ in the domain where $\mathbf{F}$ is defined.

          \textcolor{hwColor}{
            suppose $F$ is a vector field. If there exists a scalar function $f$, so that $\overrightarrow{\nabla}f=F, ~ F$ is 
            conservative and we call $f$ a potential. Assume a potential exists, let's find it. \\ \\
            $
              \overrightarrow{\nabla} \phi=F \Rightarrow ~ <\dfrac{\partial \phi}{\partial x}, \dfrac{\partial \phi}{\partial y}, \dfrac{\partial \phi}{\partial z}>=<\dfrac{zx}{\sqrt{x^2+y^2}}, \dfrac{zy}{\sqrt{x^2+y^2}}, \dfrac{1}{z^{3/4}}+\sqrt{x^2+y^2}> \\ 
              \\
              \begin{cases}
                \dfrac{\partial \phi}{\partial x}=\dfrac{zx}{\sqrt{x^2+y^2}} ~~~~ \mathbf{(A)}\\
                \\
                \dfrac{\partial \phi}{\partial y}=\dfrac{zy}{\sqrt{x^2+y^2}} ~~~~ \mathbf{(B)}\\
                \\
                \dfrac{\partial \phi}{\partial z}=\dfrac{1}{z^{3/4}}+\sqrt{x^2+y^2} ~~~~ \mathbf{(C)}\\ \\
              \end{cases} \\
            $
            The integral of $\dfrac{\partial \phi}{\partial x}$ we respect to $x$ is: \\ \\
            $
              \phi=\bigints \dfrac{zx}{\sqrt{x^2+y^2}} dx, u=x^2+y^2 \Rightarrow dx=\dfrac{du}{2x} \\
              \\
              =\dfrac{z}{2} \left[\dfrac{u^{\dfrac{1}{2}}}{\dfrac{1}{2}}\right] \Rightarrow \phi=z\sqrt{x^2+y^2}+h(y,z) \\ \\
            $
            \rule{15cm}{1pt}
            Now let's find the partial of $\phi$ with respect to $y$. \\
            \\
            $\dfrac{\partial \phi}{\partial y}=\dfrac{\partial (z\sqrt{x^2+y^2}+h(y,z))}{\partial y}=\dfrac{yz}{\sqrt{x^2+y^2}}+h_y(y,z)$ \\ \\
            By equating $\dfrac{zy}{\sqrt{x^2+y^2}}$ from $\mathbf{(B)}$ we have: \\
            \\
            $
              \dfrac{zy}{\sqrt{x^2+y^2}}=\dfrac{yz}{\sqrt{x^2+y^2}}+h_y(y,z) \Rightarrow  h_y(y,z)=0 ~~~~ \mathbf{(D)} \\
              \\
              \bigints h_y(y,z) dy=w(z) ~~~~ \mathbf{(E)}
            $ \\ \\
            Now considering $\mathbf{(D)} ~ and ~ \mathbf{(E)} \Rightarrow \phi=z\sqrt{x^2+y^2}+h(y,z)$ we can rewrite $\phi$ as: \\
            \\
            $
              \phi=z\sqrt{x^2+y^2}+w(z) \\
            $
            \rule{15cm}{1pt}
            Let's do the same thing one more time. Taking the partial derivative of $\phi$ with respect to $z$: \\
            \\
            $
              \dfrac{\partial \phi}{\partial z}=\dfrac{\partial (z\sqrt{x^2+y^2}+w(z))}{\partial z}=\sqrt{x^2+y^2}+w^{\prime}(z) \\ \\
            $
            Now it's time to equate $\mathbf{(C)}$ with what we found for $\dfrac{\partial \phi}{\partial z}$. \\
            \\
            $
              \dfrac{1}{z^{3/4}}+\sqrt{x^2+y^2}=\sqrt{x^2+y^2}+w^{\prime}(z) \Rightarrow w^{\prime}(z)=\dfrac{1}{z^{3/4}} \\
              \\
              w(z)=\bigints \dfrac{1}{z^{3/4}} dz=4z^{\dfrac{1}{4}} \\
              \\
              \\
              \Longrightarrow \phi=z\sqrt{x^2+y^2}+4z^{\dfrac{1}{4}}+C
            $
          }

      \end{enumerate}
      (Hint: follow the example shown in class.)


    \item Compute the flux of each of the following vector fields that penetrates a square area
    of side 1 centered on the x-axis at $x=1$. The square's sides are parallel to the y- and z-axes.

      \textcolor{hwColor}{
        The rate of flow (amount/volume of flow across the surface) is called the \textbf{Flux}. \\
        \\
        $F.\hat{n} ds \Rightarrow$ Vector field.Normal at any point on surface $\times$ Really small piece of surface (area) \\ \\
        $
          \bigints_{S}\bigints F.ds=\bigints_{S} \bigints F.\hat{n} ds 
        $ called the \textbf{Flux} integral. In other words \textbf{Flux}=$(F.\hat{n}) \times$ Area. The dot product basically takes
        the component of the vector field that is parallel to $\hat{n}$ and so perpendicular to
        the plane and discards the other component. Finally, we multiply by the area. \\
      }

      \begin{enumerate}
        \item $\mathbf{V}=xyz\mathbf{i}$.

          \textcolor{hwColor}{
            Imagine that the vector field is a fluid flowing through space, then the flux of the field through an area is the amount of "fluid"
            flowing through that area. Now, in this case the area we are flowing through is $L^2$ (Assume $L$ is length of the sides of the square).
            \\
            \\
            \textbf{Flux}=$\left[(xyz \hat{i}).(\hat{n})\right] L^2=\left[(xyz ~ \hat{i}).(\hat{i})\right]L^2=xyzL^2$ where $x=1$. \\
            \\
            $\therefore$ \textbf{Flux}$=yzL^2$
          }

        \item $\mathbf{V}=y\mathbf{k}+z\mathbf{j}$.

          \textcolor{hwColor}{
            \textbf{Flux}=$\left[(z \hat{j}+y\hat{k}).(\hat{n})\right] L^2=\left[(z \hat{j}+y\hat{k}).(\hat{i})\right] L^2$ \\
            \\
            $\therefore$ \textbf{Flux}$=0$
          }

        \item $\mathbf{V}=\mathbf{r}$ (where $\mathbf{r}$ is the position vector)

          \textcolor{hwColor}{
            Assume the position vector is: $r=|\mathbf{r}|=\sqrt{x^2+y^2+y^2} \Rightarrow \mathbf{r}=x \hat{i}+y \hat{j}+z \hat{k}$ \\
            \\
            \textbf{Flux}=$\left[(x \hat{i}+y \hat{j}+z \hat{k}).(\hat{n})\right] L^2=\left[(x \hat{i}+y \hat{j}+z \hat{k}).(\hat{i})\right] L^2=x L^2$ \\
            \\
            Since $x=1 ~$ $\therefore$ \textbf{Flux}=$L^2$ 
          }

      \end{enumerate}
      (Hint: Take the unit vector $\mathbf{\hat n}=\mathbf{j} \times \mathbf{k} = \mathbf{i}$ as the positive or ``outward" sense of the surface. Note that, for this exercise, $dS=dy dz$.)

    \pagebreak

    \item Take $S$ to be the (closed) surface of a sphere of radius r centered at the origin. Compute the flux penetrating it radially outward for each of the following vector fields:
      
      \textcolor{hwColor}{
        The equation of a sphere of radius r centered at the origin ($a=0, ~ b=0, ~ c=0$) is: \\
        \\
        $
          (x-a)^2+(y-b)^2+(z-c)^2=r^2 \Rightarrow x^2+y^2+z^2=r^2 \\ \\
        $ 
        For a closed surface The divergence theorem states that the surface integral of a vector field $\overrightarrow{F}=P \hat{i}+Q \hat{j}+R \hat{k}$, which is called the flux 
        through the surface, is equal to the volume integral of the divergence over the region inside the surface. \\
        \\
        $$\iiint_T \overrightarrow{\nabla}.F dv=\iiint_T \left[\dfrac{\partial P}{\partial x}+\dfrac{\partial Q}{\partial y}+\dfrac{\partial R}{\partial z}\right] dv$$
      }

      \begin{enumerate}
        \item $\mathbf{V}=\mathbf{r}$;
        
          \textcolor{hwColor}{
            $\mathbf{r}=x \hat{i}+y \hat{j}+z \hat{k}, ~~ \overrightarrow{\nabla}\mathbf{r}=1+1+1=3 \\ \\$
            \textbf{Flux}=$\iiint_T 3dv=\bigints_{\theta=0}^{\theta=2 \pi} \bigints_{\phi=0}^{\phi=\pi} \bigints_{\theta=0}^{\theta=r} 3 \rho^2 ~ sin(\phi) ~ d\rho ~ d\phi ~ d\theta$ \\ \\
            $
              =\bigints_{\theta=0}^{\theta=2 \pi} \bigints_{\phi=0}^{\phi=\pi} r^3 sin(\phi) ~ d\phi ~ d\theta=\bigints_{\theta=0}^{\theta=2 \pi}  2r^3 ~ d\theta=2r^3.2\pi \\
              \\
              \\
              \Longrightarrow 
            $
            \textbf{Flux}=$4\pi r^3$
          }

        \item $\mathbf{V}=2 \mathbf{r}/r^3$ (where $r=|\mathbf{r}|$);

          \textcolor{hwColor}{
            Behnam was here.
          }
        
        \item $\mathbf{V}=yz\mathbf{i}+zx\mathbf{j}+xy\mathbf{k}$;

          \textcolor{hwColor}{
            Behnam was here.
          }

      \end{enumerate}
      (Hint: What is the simplest way of expressing the orthogonal unit vector $\mathbf{\hat n}$ for this surface? In part (c) you will want to use Spherical Polar coordinates, but do not waste time rewriting $\mathbf{V}$. Make judicious use of scalar products.)

  \end{enumerate}

  \pagebreak

  \textbf{Part B}
  \begin{enumerate}
    \item Use Stokes' theorem to compute the surface integral of the normal component of the curl of
    $$
    \mathbf{F}= (x+y) \mathbf{i} + (y -x) \mathbf{j} + z^3 \mathbf{k}
    $$
    over the upper half of the unit sphere (i.e. the sphere of radius 1 centered at the origin.) \\
    (Hint: The curve in question is the unit circle in the $x$-$y$ plane. Use the azimuthal angle, $\phi$, to describe the circle and the integrand.)

      \textcolor{hwColor}{
        Stokes' theorem is a generalization of Green's theorem. This theorem is used when a curve is not on a plane.  \\
        \\
        $
          \oint_C \overrightarrow{F}.d\overrightarrow{r}=\iint_S ~ curl ~ \overrightarrow{F}.d\overrightarrow{S}=\iint_S ~ curl ~ \overrightarrow{F}.\hat{n} ~ ds \\
          \\
          curl ~ \overrightarrow{F}=\overrightarrow{\nabla} \times \overrightarrow{F}=\begin{vmatrix}
            \hat{i} & \hat{j} & \hat{k} \\
            \\
            \dfrac{\partial}{\partial x} & \dfrac{\partial}{\partial y} & \dfrac{\partial}{\partial z} \\
            \\
            x+y & y-x & z^3 
          \end{vmatrix}=\hat{i}(0-0)-\hat{j}(0-0)+\hat{k}(-1-1)=-2\hat{k} \\
          \\
          \\
          \Rightarrow curl ~ \overrightarrow{F}=-2\hat{k} \\ \\
        $
        The equation of the unit sphere of radius r centered at the origin is $x^2+y^2+z^2=1$ and the upper half of the unit sphere is
        where $z\geq 0$. \\
        \\
        Normal vectors to the surface be pointing outwards at any particular point and that satisfies the right hand rule that
        when you take your finger and go along the boundary the normal would be sticking upwards which in this case is $\hat{k}$. \\ \\  
        $
          \iint_S ~ curl ~ \overrightarrow{F}.\overrightarrow{n} ~ ds \\ \\
          \\
          \overrightarrow{n} ds=\dfrac{\overrightarrow{\nabla}(x^2+y^2+z^2)}{|\overrightarrow{\nabla}(x^2+y^2+z^2)\hat{k}|}ds=\dfrac{2x \hat{i}+2y\hat{j}+2z\hat{z}}{|2z|} \\
          \\
          \iint_S ~ curl ~ \overrightarrow{F}.\overrightarrow{n} ~ ds=\iint_R ~ \left[(-2\hat{k}).(\dfrac{2x \hat{i}+2y\hat{j}+2z\hat{k}}{|2z|})\right]dA=\iint_R ~ -\dfrac{4z}{|2z|}dA=\iint_R ~ -2dA \\ \\
        $
        The region $R$ is the portion above the circle of unit 1 and it's got an area of $\pi r^2=\pi (1)^2=\pi$, hence: \\
        \\
        $
          \Longrightarrow \iint_R ~ -\dfrac{4z}{|2z|}dA=\iint_R ~ -2dA=-2\pi
        $
      }

    \pagebreak

    \item The differential form of Maxwell's third equation (Ampere's law) is (leaving out the displacement current)
    $\mathbf{\nabla}\times \mathbf{B} = \mu_0\mathbf{J}$,
    where $\mathbf{J}$ is the current density. Use Stokes' theorem to rewrite this in the well-known integral form.\\
    (Hint: The ``well-known integral form" of Ampere's law  is  $\int_{\mathcal C} \mathbf{B}\cdot d\mathbf{s} = \mu_0 I$, where $I$ is the net current linked by a closed curve ${\mathcal C}$; refer to your introductory physics books.)

      \textcolor{hwColor}{
        $
          \oint_C \overrightarrow{F}.d\overrightarrow{r}=\iint_S ~ curl ~ \overrightarrow{F}.d\overrightarrow{S} \\ \\
          \iint_S curl ~ \overrightarrow{B}.d\overrightarrow{S}=\oint \overrightarrow{B}.d\overrightarrow{l} \\
          \\
          \\
          \Rightarrow \iint_S \mu_0 \overrightarrow{J}.d\overrightarrow{S}=\oint \overrightarrow{B}.d\overrightarrow{l} \\ \\ \\
        $
        We know that $\iint_S \overrightarrow{J}.d\overrightarrow{S}$ gives us $I$. Hence, we have: \\ \\
        $\oint \overrightarrow{B}.d\overrightarrow{l}=\mu_0 ~ I$ which is the “well-known integral form” of Ampere’s law.
      }

    \item Using Gauss theorem, show that, if $\Phi$ is a scalar field,
    $$        
    \int_{\mathcal S}\Phi d{\mathbf S}=\int_{V} (\mathbf{\nabla} \Phi) dV .
    $$
    (the integral on the left side is over a closed surface, $\mathcal S$; the integral on the right side is over the volume $V$ enclosed by the surface  $\mathcal S$). \\
    (Hint: Apply Gauss theorem to the vector $\mathbf{A} = \Phi\mathbf{C}$, where $\mathbf{C}$ is an arbitrary constant vector, and use one of the properties of Table 10.1 of the textbook.)

      \textcolor{hwColor}{
        Two vectors $\overrightarrow{a}$ and $\overrightarrow{b} \in R^n$ are equal to each other if and only if for all vectors $\overrightarrow{q} \in R^n$, 
        $\overrightarrow{q}.\overrightarrow{a}=\overrightarrow{q}.\overrightarrow{b}$ \\ \\ 
        For any constant $C$, we have: \\ \\
        $
          \overrightarrow{k}\left[\bigints \overrightarrow{\nabla}\phi dv \right]=\bigints \overrightarrow{\nabla}.(\phi \overrightarrow{k})dv=\bigints \phi \overrightarrow{k}.dS=\overrightarrow{k}\left[\bigints \phi dS\right] \\ \\
        $
        The first equality holds since $\overrightarrow{k}.\overrightarrow{\nabla}\phi=\overrightarrow{\nabla}.(\phi \overrightarrow{k})-\phi(\overrightarrow{\nabla}.\overrightarrow{k})$. Moreover, $\overrightarrow{k}$
        is a constant vector, $\overrightarrow{\nabla}.\overrightarrow{k}=0$. Hence, $\overrightarrow{k}.\overrightarrow{\nabla}\phi=\overrightarrow{\nabla}.(\phi \overrightarrow{k})$. Therefore, \\ \\ \\
        $
          \bigints \overrightarrow{\nabla}\phi dv=\bigints \phi ds
        $
      }
      
  \end{enumerate}

\end{document}
