\documentclass[fleqn]{article}
\oddsidemargin 0.0in
\textwidth 6.0in
\thispagestyle{empty}
\usepackage{import}
\usepackage{amsmath}
\usepackage{graphicx}
\usepackage{flexisym}
\usepackage{amssymb}
\usepackage{bigints} 
\usepackage[english]{babel}
\usepackage[utf8x]{inputenc}
\usepackage{float}
\usepackage[colorinlistoftodos]{todonotes}

\definecolor{hwColor}{HTML}{AD53BA}

\begin{document}

  \begin{titlepage}

    \newcommand{\HRule}{\rule{\linewidth}{0.5mm}} % Defines a new command for the horizontal lines, change thickness here

    \center % Center everything on the page



    \textsc{\LARGE Arizona State University}\\[1.5cm] % Name of your university/college

    \textsc{\LARGE Mathematical Methods For Physics II }\\[1.5cm] % Major heading such as course name


    \begin{figure}
      \includegraphics[width=\linewidth]{asu.png}
    \end{figure}


    \HRule \\[0.4cm]
    { \huge \bfseries Homework Four}\\[0.4cm] 
    \HRule \\[1.5cm]

    \textbf{Behnam Amiri}

    \bigbreak

    \textbf{Prof: Cecilia Lunardini}

    \bigbreak


    \textbf{{\large \today}\\[2cm]}

    \vfill % Fill the rest of the page with whitespace

  \end{titlepage}

  \begin{enumerate}
    \item Consider the vector field 
      $$
      \mathbf{F}= \left(\frac{zx}{\sqrt{x^2 + y^2} }  , \frac{zy}{\sqrt{x^2 + y^2}}, \frac{1}{z^{3/4}} + \sqrt{x^2 + y^2} \right)~
      $$
      defined for $x\neq 0$, $y\neq 0$ and $z>0$. 
      \begin{enumerate}
        \item Compute the curl of $\mathbf{F}$.

          \textcolor{hwColor}{
            The curl of a vector field measures the rotation or spinning effect. \\
            \\
            $
              curl ~ F(x,y,z)=\overrightarrow{\nabla} ~ \times ~ F(x,y,z)=\begin{vmatrix}
                \hat{i} & \hat{j} & \hat{k} \\
                \\
                \dfrac{\partial}{\partial x} & \dfrac{\partial}{\partial y} & \dfrac{\partial}{\partial z} \\
                \\
                \dfrac{zx}{\sqrt{x^2+y^2}} & \dfrac{zy}{\sqrt{x^2+y^2}} & \dfrac{1}{z^{3/4}}+\sqrt{x^2+y^2}
              \end{vmatrix}
              \\
              \\
              \\
              =\hat{i}\left[\dfrac{\partial}{\partial y}(\dfrac{1}{z^{3/4}}+\sqrt{x^2+y^2})-\dfrac{\partial}{\partial z}(\dfrac{zy}{\sqrt{x^2+y^2}})\right] \\ \\   
              -\hat{j}\left[\dfrac{\partial}{\partial x}(\dfrac{1}{z^{3/4}}+\sqrt{x^2+y^2})-\dfrac{\partial}{\partial z}(\dfrac{zx}{\sqrt{x^2+y^2}})\right] \\ \\
              +\hat{k}\left[\dfrac{\partial}{\partial x}(\dfrac{zy}{\sqrt{x^2+y^2}})-\dfrac{\partial}{\partial y}(\dfrac{zx}{\sqrt{x^2+y^2}})\right] \\ \\
              \\
              \\
              =\hat{i}\left[\dfrac{y}{\sqrt{x^2+y^2}}-\dfrac{y}{\sqrt{x^2+y^2}}\right] \\ \\
              -\hat{j}\left[\dfrac{x}{\sqrt{x^2+y^2}}-\dfrac{x}{\sqrt{x^2+y^2}}\right] \\ \\
              +\hat{k}\left[-\dfrac{xyz}{(x^2+y^2)^{3/2}}+\dfrac{xyz}{(x^2+y^2)^{3/2}}\right] \\ \\
              \\
              \overrightarrow{\nabla} ~ \times ~ F(x,y,z)=0 \\ \\
            $
            Since $\mathbf{curl ~ F=0}$, then we can say $\mathbf{F}$ is conservative.
          } 

        \item Find a potential field corresponding to $\mathbf{F}$, i.e., a scalar field $\phi$ such that $\mathbf{F}=\mathbf{\nabla}\phi$ in the domain where $\mathbf{F}$ is defined.

          \textcolor{hwColor}{
            suppose $F$ is a vector field. If there exists a scalar function $f$, so that $\overrightarrow{\nabla}f=F, ~ F$ is 
            conservative and we call $f$ a potential. Assume a potential exists, let's find it. \\ \\
            $
              \overrightarrow{\nabla} \phi=F \Rightarrow ~ <\dfrac{\partial \phi}{\partial x}, \dfrac{\partial \phi}{\partial y}, \dfrac{\partial \phi}{\partial z}>=<\dfrac{zx}{\sqrt{x^2+y^2}}, \dfrac{zy}{\sqrt{x^2+y^2}}, \dfrac{1}{z^{3/4}}+\sqrt{x^2+y^2}> \\ 
              \\
              \begin{cases}
                \dfrac{\partial \phi}{\partial x}=\dfrac{zx}{\sqrt{x^2+y^2}} ~~~~ \mathbf{(A)}\\
                \\
                \dfrac{\partial \phi}{\partial y}=\dfrac{zy}{\sqrt{x^2+y^2}} ~~~~ \mathbf{(B)}\\
                \\
                \dfrac{\partial \phi}{\partial z}=\dfrac{1}{z^{3/4}}+\sqrt{x^2+y^2} ~~~~ \mathbf{(C)}\\ \\
              \end{cases} \\
            $
            The integral of $\dfrac{\partial \phi}{\partial x}$ we respect to $x$ is: \\ \\
            $
              \phi=\bigints \dfrac{zx}{\sqrt{x^2+y^2}} dx, u=x^2+y^2 \Rightarrow dx=\dfrac{du}{2x} \\
              \\
              =\dfrac{z}{2} \left[\dfrac{u^{\dfrac{1}{2}}}{\dfrac{1}{2}}\right] \Rightarrow \phi=z\sqrt{x^2+y^2}+h(y,z) \\ \\
            $
            \rule{15cm}{1pt}
            Now let's find the partial of $\phi$ with respect to $y$. \\
            \\
            $\dfrac{\partial \phi}{\partial y}=\dfrac{\partial (z\sqrt{x^2+y^2}+h(y,z))}{\partial y}=\dfrac{yz}{\sqrt{x^2+y^2}}+h_y(y,z)$ \\ \\
            By equating $\dfrac{zy}{\sqrt{x^2+y^2}}$ from $\mathbf{(B)}$ we have: \\
            \\
            $
              \dfrac{zy}{\sqrt{x^2+y^2}}=\dfrac{yz}{\sqrt{x^2+y^2}}+h_y(y,z) \Rightarrow  h_y(y,z)=0 ~~~~ \mathbf{(D)} \\
              \\
              \bigints h_y(y,z) dy=w(z) ~~~~ \mathbf{(E)}
            $ \\ \\
            Now considering $\mathbf{(D)} ~ and ~ \mathbf{(E)} \Rightarrow \phi=z\sqrt{x^2+y^2}+h(y,z)$ we can rewrite $\phi$ as: \\
            \\
            $
              \phi=z\sqrt{x^2+y^2}+w(z) \\
            $
            \rule{15cm}{1pt}
            Let's do the same thing one more time. Taking the partial derivative of $\phi$ with respect to $z$: \\
            \\
            $
              \dfrac{\partial \phi}{\partial z}=\dfrac{\partial (z\sqrt{x^2+y^2}+w(z))}{\partial z}=\sqrt{x^2+y^2}+w^{\prime}(z) \\ \\
            $
            Now it's time to equate $\mathbf{(C)}$ with what we found for $\dfrac{\partial \phi}{\partial z}$. \\
            \\
            $
              \dfrac{1}{z^{3/4}}+\sqrt{x^2+y^2}=\sqrt{x^2+y^2}+w^{\prime}(z) \Rightarrow w^{\prime}(z)=\dfrac{1}{z^{3/4}} \\
              \\
              w(z)=\bigints \dfrac{1}{z^{3/4}} dz=4z^{\dfrac{1}{4}} \\
              \\
              \\
              \Longrightarrow \phi=z\sqrt{x^2+y^2}+4z^{\dfrac{1}{4}}+C
            $
          }

      \end{enumerate}
      (Hint: follow the example shown in class.)


    \item Compute the flux of each of the following vector fields that penetrates a square area
    of side 1 centered on the x-axis at $x=1$. The square's sides are parallel to the y- and z-axes.

      \textcolor{hwColor}{
        The rate of flow (amount/volume of flow across the surface) is called the \textbf{Flux}. \\
        \\
        $F.\hat{n} ds \Rightarrow$ Vector field.Normal at any point on surface $\times$ Really small piece of surface (area) \\ \\
        $
          \bigints_{S}\bigints F.ds=\bigints_{S} \bigints F.\hat{n} ds 
        $ called the \textbf{Flux} integral. In other words \textbf{Flux}=$(F.\hat{n}) \times$ Area. The dot product basically takes
        the component of the vector field that is parallel to $\hat{n}$ and so perpendicular to
        the plane and discards the other component. Finally, we multiply by the area. \\
      }

      \begin{enumerate}
        \item $\mathbf{V}=xyz\mathbf{i}$.

          \textcolor{hwColor}{
            Imagine that the vector field is a fluid flowing through space, then the flux of the field through an area is the amount of "fluid"
            flowing through that area. Now, in this case the area we are flowing through is $L^2$ (Assume $L$ is length of the sides of the square).
            \\
            \\
            \textbf{Flux}=$\left[(xyz \hat{i}).(\hat{n})\right] L^2=\left[(xyz ~ \hat{i}).(\hat{i})\right]L^2=xyzL^2$ where $x=1$. \\
            \\
            $\therefore$ \textbf{Flux}$=yzL^2$
          }

        \item $\mathbf{V}=y\mathbf{k}+z\mathbf{j}$.

          \textcolor{hwColor}{
            \textbf{Flux}=$\left[(z \hat{j}+y\hat{k}).(\hat{n})\right] L^2=\left[(z \hat{j}+y\hat{k}).(\hat{i})\right] L^2$ \\
            \\
            $\therefore$ \textbf{Flux}$=0$
          }

        \item $\mathbf{V}=\mathbf{r}$ (where $\mathbf{r}$ is the position vector)

          \textcolor{hwColor}{
            Assume the position vector is: $r=|\mathbf{r}|=\sqrt{x^2+y^2+y^2} \Rightarrow \mathbf{r}=x \hat{i}+y \hat{j}+z \hat{k}$ \\
            \\
            \textbf{Flux}=$\left[(x \hat{i}+y \hat{j}+z \hat{k}).(\hat{n})\right] L^2=\left[(x \hat{i}+y \hat{j}+z \hat{k}).(\hat{i})\right] L^2=x L^2$ \\
            \\
            Since $x=1 ~$ $\therefore$ \textbf{Flux}=$L^2$ 
          }

      \end{enumerate}
      (Hint: Take the unit vector $\mathbf{\hat n}=\mathbf{j} \times \mathbf{k} = \mathbf{i}$ as the positive or ``outward" sense of the surface. Note that, for this exercise, $dS=dy dz$.)

    \pagebreak

    \item Take $S$ to be the (closed) surface of a sphere of radius r centered at the origin. Compute the flux penetrating it radially outward for each of the following vector fields:
      \begin{enumerate}
        \item $\mathbf{V}=\mathbf{r}$ ;
        \item $\mathbf{V}=2 \mathbf{r}/r^3$ (where $r=|\mathbf{r}|$); 
        \item $\mathbf{V}=yz\mathbf{i}+zx\mathbf{j}+xy\mathbf{k}$.
      \end{enumerate}
      (Hint: What is the simplest way of expressing the orthogonal unit vector $\mathbf{\hat n}$ for this surface? In part (c) you will want to use Spherical Polar coordinates, but do not waste time rewriting $\mathbf{V}$. Make judicious use of scalar products.)

  \end{enumerate}

\end{document}
