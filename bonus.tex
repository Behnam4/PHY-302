\documentclass[fleqn]{article}
\oddsidemargin 0.0in
\textwidth 6.0in
\thispagestyle{empty}
\usepackage{import}
\usepackage{amsmath}
\usepackage{graphicx}
\usepackage{flexisym}
\usepackage{amssymb}
\usepackage{bigints} 
\usepackage[english]{babel}
\usepackage[utf8x]{inputenc}
\usepackage{float}
\usepackage[colorinlistoftodos]{todonotes}

\definecolor{hwColor}{HTML}{AD53BA}

\begin{document}

  \begin{titlepage}

    \newcommand{\HRule}{\rule{\linewidth}{0.5mm}}

    \center 


    \textsc{\LARGE Arizona State University}\\[1.5cm]

    \textsc{\LARGE Mathematical Methods For Physics II }\\[1.5cm]


    \begin{figure}
      \includegraphics[width=\linewidth]{asu.png}
    \end{figure}


    \HRule \\[0.4cm]
    { \huge \bfseries Bonus Exercise}\\[0.4cm] 
    \HRule \\[1.5cm]

    \textbf{Behnam Amiri}

    \bigbreak

    \textbf{Prof: Cecilia Lunardini}

    \bigbreak


    \textbf{{\large \today}\\[2cm]}

    \vfill

  \end{titlepage}

    Reconsider the example on branch cuts shown in class 
    (See class 23 Module and page 836 of the textbook) and discuss the two possible choices of branch cuts, shown in fig. 
    24.2 (b) and 24.2 (c) of the book. Show that, for appropriate (different) choices of the intervals for the phases 
    of $z-i$ and $z+i$ (e.g., $\theta_1 \in [ -\pi/2 , 3 \pi/2]$), the points along the branch cuts correspond to the 
    points where the function is not single valued (i.e., the limit of the function from one side of the cut is different 
    from the limit of the function from the other side of the cut). Computer graphics or high quality, creative by-hand graphics is strongly encouraged. 

    \textcolor{hwColor}{
      \\
      \\
      \\
      Behnam was here
    }

\end{document}
