\documentclass[fleqn]{article}
\oddsidemargin 0.0in
\textwidth 6.0in
\thispagestyle{empty}
\usepackage{import}
\usepackage{amsmath}
\usepackage{graphicx}
\usepackage{flexisym}
\usepackage{amssymb}
\usepackage{bigints} 
\usepackage[english]{babel}
\usepackage[utf8x]{inputenc}
\usepackage{float}
\usepackage[colorinlistoftodos]{todonotes}

\definecolor{hwColor}{HTML}{AD53BA}

\begin{document}

  \begin{titlepage}

    \newcommand{\HRule}{\rule{\linewidth}{0.5mm}}

    \center 


    \textsc{\LARGE Arizona State University}\\[1.5cm]

    \textsc{\LARGE Mathematical Methods For Physics II }\\[1.5cm]


    \begin{figure}
      \includegraphics[width=\linewidth]{asu.png}
    \end{figure}


    \HRule \\[0.4cm]
    { \huge \bfseries Bonus Exercise}\\[0.4cm] 
    \HRule \\[1.5cm]

    \textbf{Behnam Amiri}

    \bigbreak

    \textbf{Prof: Cecilia Lunardini}

    \bigbreak


    \textbf{{\large \today}\\[2cm]}

    \vfill

  \end{titlepage}

    Reconsider the example on branch cuts shown in class 
    (See class 23 Module and page 836 of the textbook) and discuss the two possible choices of branch cuts, shown in fig. 
    24.2 (b) and 24.2 (c) of the book. Show that, for appropriate (different) choices of the intervals for the phases 
    of $z-i$ and $z+i$ (e.g., $\theta_1 \in [ -\pi/2 , 3 \pi/2]$), the points along the branch cuts correspond to the 
    points where the function is not single valued (i.e., the limit of the function from one side of the cut is different 
    from the limit of the function from the other side of the cut). Computer graphics or high quality, creative by-hand graphics is strongly encouraged. 

    \textcolor{hwColor}{
      \\
      \\
      \\
      Let's define branch cuts first. The regions where a function is not analytic are
      called singularities. One type of singularity involves regions or points where
      the function is not single-valued; these are called branch point. \\ \\
      From the page 836 of the textbook we have $f(z)=\sqrt{z^2+1}=\sqrt{(z+i)(z-i)}$. \\
      \\
      \\
      $
        f(z)=\sqrt{(z+i)(z-i)}=\sqrt{r_1 r_2 e^{i \theta_1} ~ e^{i \theta_2}}
      $
      \\
      \\
      \rule{15cm}{1pt}
      \\
      \\
      Rotation around $i$: \\
      \\
      $
        f(z)=\sqrt{r_1 r_2} e^{i\dfrac{(\theta_1+\theta_2)}{2}}=\sqrt{r_1 r_2} e^{i\dfrac{(\theta_1+\theta_2+2 \pi)}{2}} \\
        \\
        \\
        \therefore ~~~ f(z)=-f(z) ~~~ \surd \\
      $
      \\
      The value of the function changes with a Rotation therefore $-i$ and $i$ are a branch points. 
      \\
      \\
      \rule{15cm}{1pt}
      \\
      \\
      For complete a circle: \\
      \\
      $
        f(z)=\sqrt{r_1 r_2} e^{i\dfrac{(\theta_1+\theta_2)}{2}}=\sqrt{r_1 r_2} e^{i\dfrac{(\theta_1+\theta_2+4 \pi)}{2}}
        =\sqrt{r_1 r_2} ~ e^{i2 \pi} e^{i\dfrac{(\theta_1+\theta_2)}{2}}=f(z)
      $
      \\
      \\
      What just happened is that the path of the circle does NOT go over the finite branch cut. $[i,-i]$ \\
      \\
      We have the following cases: \\
      \\
      \begin{itemize}
        \item Rotation around $i$ counterclockwise. 
        \item Rotation around $i$ clockwise.
        \item Rotation around $-i$ counterclockwise. 
        \item Rotation around $-i$ clockwise.
      \end{itemize}.
      \\
      \\
      \rule{15cm}{1pt}
      \\
      \\
      $
        \theta_1=\dfrac{3 \pi}{2}, ~~~ \theta_2=0 \\
        \\
        \\
        f(z)=\sqrt{r_1 r_2} e^{i\dfrac{(\theta_1+\theta_2)}{2}}=\sqrt{r_1 r_2} e^{i\dfrac{3 \pi}{4}} \\
        \\
        f(z)=(-1)^{\dfrac{3}{4}} \sqrt{r_1 r_2}
      $
      \\
      \\
      \rule{15cm}{1pt}
      \\
      \\
      $
        \theta_1=-\dfrac{\pi}{2}, ~~~ \theta_2=0 \\
        \\
        \\
        f(z)=\sqrt{r_1 r_2} e^{i\dfrac{(\theta_1+\theta_2)}{2}}=\sqrt{r_1 r_2} e^{-i\dfrac{\pi}{4}} \\
        \\
        f(z)=(-1)^{\dfrac{3}{4}} \sqrt{r_1 r_2}
      $
      \\
      \\
      \rule{15cm}{1pt}
      \\
      \\
      $
        \theta_1=0, ~~~ \theta_2=\dfrac{\pi}{2} \\
        \\
        \\
        f(z)=\sqrt{r_1 r_2} e^{i\dfrac{(\theta_1+\theta_2)}{2}}=\sqrt{r_1 r_2} e^{i\dfrac{\pi}{4}} \\
        \\
        f(z)=(-1)^{\dfrac{1}{4}} \sqrt{r_1 r_2}
      $
      \\
      \\
      \rule{15cm}{1pt}
      \\
      \\
      $
        \theta_1=0, ~~~ \theta_2=-\dfrac{3\pi}{2} \\
        \\
        \\
        f(z)=\sqrt{r_1 r_2} e^{i\dfrac{(\theta_1+\theta_2)}{2}}=\sqrt{r_1 r_2} e^{-i\dfrac{3\pi}{4}} \\
        \\
        f(z)=-(-1)^{\dfrac{1}{4}} \sqrt{r_1 r_2}
      $
    }

\end{document}
