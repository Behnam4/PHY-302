\documentclass[fleqn]{article}
\oddsidemargin 0.0in
\textwidth 6.0in
\thispagestyle{empty}
\usepackage{import}
\usepackage{amsmath}
\usepackage{graphicx}
\usepackage{flexisym}
\usepackage{amssymb}
\usepackage{bigints} 
\usepackage[english]{babel}
\usepackage[utf8x]{inputenc}
\usepackage{float}
\usepackage[colorinlistoftodos]{todonotes}

\definecolor{hwColor}{HTML}{AD53BA}

\begin{document}

  \begin{titlepage}

    \newcommand{\HRule}{\rule{\linewidth}{0.5mm}} % Defines a new command for the horizontal lines, change thickness here

    \center % Center everything on the page



    \textsc{\LARGE Arizona State University}\\[1.5cm] % Name of your university/college

    \textsc{\LARGE Mathematical Methods For Physics II }\\[1.5cm] % Major heading such as course name


    \begin{figure}
      \includegraphics[width=\linewidth]{asu.png}
    \end{figure}


    \HRule \\[0.4cm]
    { \huge \bfseries Homework Eleven}\\[0.4cm] 
    \HRule \\[1.5cm]

    \textbf{Behnam Amiri}

    \bigbreak

    \textbf{Prof: Cecilia Lunardini}

    \bigbreak


    \textbf{{\large \today}\\[2cm]}

    \vfill % Fill the rest of the page with whitespace

  \end{titlepage}

  \textbf{Part A}
  \begin{enumerate}

    \item  Prove the \emph{completeness relationship}, eq. (18.51) of the textbook. (Hint: treat the Dirac Delta like you would treat a generic function, and write it in the basis of the Spherical Harmonics). 
    
    
    \item  Prove that the differential equation that has $\sin( k x)$ and $\cos(k x)$ as solutions (with $k$ a constant) is a Sturm-Liouville equation. 
    
    
    \item Consider a system of three point-like electric charges, with charges $q_i$ and positions $\vec r_i$ ($i=1,2,3$), as follows:
    % (see figure): 
    $q_1=+Q$, $\vec r_1=(0,0,d)$; $q_2=+Q$, $\vec r_2=(d,0,0)$; $q_3=-Q$, $\vec r_3=(0,0,0)$.   
    
    \begin{enumerate}
    \item Calculate the ``outer" monopole moment of the given system of charges. 
    
    \item Calculate the ``outer" dipole moment for $m=0$. 
    
    
    \end{enumerate}
    (Hint: follow the example shown in class). 
    
  \end{enumerate}

  \textbf{Part B}
  \begin{enumerate}

    \item Using the definition of analytic function, determine which of the complex functions below are analytic: 
    
    \begin{enumerate}
      \item $F(x,y)=x^2+2ixy+y^2$;
      \item  $F(x,y)=x^2-y^2$;
      \item  $F(x,y)=x^2-2ixy-y^2$;
      \item  $F(x,y)=x^2+2ixy-y^2$;
      \item  $F(x,y)=e^x(\cos y+i \sin y)$;
      \item  $F(x,y)=e^y(\cos x+i \sin x)$.
    \end{enumerate}
    
    \item Apply the Cauchy Riemann conditions to the complex functions in the Exercise 1 above.
    
    
  \end{enumerate}

\end{document}
