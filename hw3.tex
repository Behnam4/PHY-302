\documentclass[fleqn]{article}
\oddsidemargin 0.0in
\textwidth 6.0in
\thispagestyle{empty}
\usepackage{import}
\usepackage{amsmath}
\usepackage{amssymb}
\usepackage{graphicx}
\usepackage{bigints} 
\usepackage[english]{babel}
\usepackage[utf8x]{inputenc}
\usepackage{float}
\usepackage[colorinlistoftodos]{todonotes}
\usepackage{mathtools}
\usepackage[thinc]{esdiff}

\definecolor{hwColor}{HTML}{C14A96}

\begin{document}

  \begin{titlepage}

    \newcommand{\HRule}{\rule{\linewidth}{0.5mm}}

    \center

    \textsc{\LARGE Arizona State University}\\[1.5cm]

    \textsc{\LARGE Mathematical Methods For Physics II }\\[1.5cm]


    \begin{figure}
      \includegraphics[width=\linewidth]{asu.png}
    \end{figure}


    \HRule \\[0.4cm]
    { \huge \bfseries Homework Three}\\[0.4cm] 
    \HRule \\[1.5cm]

    \textbf{Behnam Amiri}

    \bigbreak

    \textbf{Prof: Cecilia Lunardini}

    \bigbreak


    \textbf{{\large \today}\\[2cm]}

    \vfill

  \end{titlepage}

  \textbf{Part A}
  \begin{enumerate}
    \item Using the general formula for the gradient in generalized curvilinear coordinates, derive the expression for the gradient in cylindrical polar coordinates (shown in table 10.2 of the textbook)

    \textcolor{hwColor}{
      Relationship between cartesian and cylindrical coordinates: \\
      \\
      $
        \begin{cases}
          x=\rho ~ cos(\phi) \\
          y=\rho ~ sin(\phi) \\
          z=z
        \end{cases}
        \Longleftrightarrow \begin{cases}
          \rho = \sqrt{x^2+y^2} \\
          \phi = arctan(\dfrac{y}{x}) \\
          z=z
        \end{cases} \\
        \\
        \\
        \overrightarrow{r}=x \hat{i}+y \hat{j}+z \hat{k}=\rho ~ cos(\phi) \hat{i}+\rho ~ sin(\phi) \hat{j}+z \hat{k} \\
        \\
        \overrightarrow{dr}=\dfrac{\partial \overrightarrow{r}}{\partial \rho} ~ d\rho+\dfrac{\partial \overrightarrow{r}}{\partial \phi} ~ d\phi+\dfrac{\partial \overrightarrow{r}}{\partial z} ~ dz=d\rho ~ e_{\rho}+d\phi ~ e_{\phi}+dz ~ e_z=d\rho ~ \hat{e}_{\rho}+\rho ~ d\phi ~ \hat{e}_{\phi}+dz ~ \hat{e_z} \\
        \\
        \\
        \begin{cases}
          \hat{e}_{\rho}=e_{\rho}=\dfrac{\dfrac{\partial \overrightarrow{r}}{\partial \rho}}{|\dfrac{\partial \overrightarrow{r}}{\partial \rho}|}=cos(\phi) ~ \hat{i}+sin(\phi) ~ \hat{j} \\
          \\
          \hat{e}_{\phi}=\dfrac{1}{\rho}e_{\phi}=\dfrac{\dfrac{\partial \overrightarrow{r}}{\partial \phi}}{|\dfrac{\partial \overrightarrow{r}}{\partial \phi}|}=-sin(\phi) ~ \hat{i}+cos(\phi) ~ \hat{j}  \\
          \\
          \hat{e}_z=e_z=\dfrac{\dfrac{\partial \overrightarrow{r}}{\partial z}}{|\dfrac{\partial \overrightarrow{r}}{\partial z}|}=\hat{k}  \\
        \end{cases} \\
        \\
        \dfrac{\partial \rho}{\partial y}=\dfrac{y}{\sqrt{x^2+y^2}}=\dfrac{\rho ~ sin(\phi)}{\rho}=sin(\phi), ~~
        \dfrac{\partial \rho}{\partial x}=\dfrac{x}{\sqrt{x^2+y^2}}=\dfrac{\rho ~ cos(\phi)}{\rho}=cos(\phi) \\
        \\
        \\
        \dfrac{\partial \phi}{\partial x}=-\dfrac{y}{x^2}\dfrac{1}{1+(\dfrac{y}{x})^2}=-\dfrac{y}{x^2+y^2}=-\dfrac{\rho ~ sin(\phi)}{\rho^2}=-\dfrac{sin(\phi)}{\rho} \\
        \\
        \dfrac{\partial \phi}{\partial y}=\dfrac{1}{x}\dfrac{1}{1+(\dfrac{y}{x})^2}=\dfrac{x}{x^2+y^2}=\dfrac{\rho ~ cos(\phi)}{\rho^2}=\dfrac{cos(\phi)}{\rho} \\
      $ 
    }

    \pagebreak

    \textcolor{hwColor}{
      The unit vector conversation from Cartesian to Cylindrical: \\
      \\
      $
        \hat{i}=cos(\phi) ~ \hat{e}_{\rho}-sin(\phi) ~ \hat{e}_{\phi}, ~~~~ \hat{j}=sin(\phi) ~ \hat{e}_{\rho}+cos(\phi) ~ \hat{e}_{\phi}
        \\
        \\
        \dfrac{\partial }{\partial x}=\dfrac{\partial }{\partial \rho}\dfrac{\partial \rho}{\partial x}+\dfrac{\partial }{\partial \phi}\dfrac{\partial \phi}{\partial x}, ~~~ \dfrac{\partial }{\partial y}=\dfrac{\partial }{\partial \rho}\dfrac{\partial \rho}{\partial y}+\dfrac{\partial }{\partial \phi}\dfrac{\partial \phi}{\partial y} \\
        \\
        \overrightarrow{\nabla}=\dfrac{\partial}{\partial x} \hat{i}+\dfrac{\partial}{\partial y}\hat{j}+\dfrac{\partial}{\partial z}\hat{k} \\
        \\
        \\
        =\left(\dfrac{\partial }{\partial \rho}\dfrac{\partial \rho}{\partial x}+\dfrac{\partial }{\partial \phi}\dfrac{\partial \phi}{\partial x}\right)\left(cos(\phi) ~ \hat{e}_{\rho}-sin(\phi) ~ \hat{e}_{\phi}\right) \\
        +\left(\dfrac{\partial }{\partial \rho}\dfrac{\partial \rho}{\partial y}+\dfrac{\partial }{\partial \phi}\dfrac{\partial \phi}{\partial y}\right)\left(sin(\phi) ~ \hat{e}_{\rho}+cos(\phi) ~ \hat{e}_{\phi}\right) \\
        +\dfrac{\partial}{\partial z}\hat{e}_z
        \\
        \\
        \\
        \\
        =\left(\dfrac{\partial}{\partial \rho} cos(\phi)+\dfrac{\partial}{\partial \phi}\dfrac{-sin(\phi)}{\rho}\right)\left(cos(\phi) ~ \hat{e}_{\rho}-sin(\phi) ~ \hat{e}_{\phi}\right) \\
        +\left(\dfrac{\partial}{\partial \rho} sin(\phi)+\dfrac{\partial}{\partial \phi}\dfrac{cos(\phi)}{\rho}\right)\left(sin(\phi) ~ \hat{e}_{\rho}+cos(\phi) ~ \hat{e}_{\phi}\right) \\
        +\dfrac{\partial}{\partial z}\hat{e}_z \\
        \\
        \\
        \\
        \\
        =cos^2(\phi) \dfrac{\partial}{\partial \rho}\hat{e}_{\rho}-cos(\phi) ~ sin(\phi) \dfrac{\partial}{\partial \rho} \hat{e}_{\phi}-\dfrac{sin(\phi) ~ cos(\phi)}{\rho} \dfrac{\partial}{\partial \phi}\hat{e}_{\rho}+\dfrac{sin^2(\phi)}{\rho} \dfrac{\partial}{\partial \phi} \hat{e}_{\phi} \\
        +sin^2(\phi) \dfrac{\partial}{\partial \rho}\hat{e}_{\rho}+sin(\phi) ~ cos(\phi) \dfrac{\partial}{\partial \rho} \hat{e}_{\phi}+\dfrac{cos(\phi) ~ sin(\phi)}{\rho} \dfrac{\partial}{\partial \phi}\hat{e}_{\rho}+\dfrac{cos^2(\phi)}{\rho}\dfrac{\partial}{\partial \phi}\hat{e}_{\phi} \\
        \\
        \\
        \\
        \\
        =(sin^2(\phi)+cos^2(\phi))\dfrac{\partial}{\partial \rho}\hat{e}_{\rho}+(sin^2(\phi)+cos^2(\phi))\dfrac{1}{\rho}\dfrac{\partial}{\partial \phi}\hat{e}_{\phi}+\dfrac{\partial}{\partial z}\hat{e}_z \\
        \\
        \\
        \\
        \Longrightarrow \overrightarrow{\nabla}=\dfrac{\partial}{\partial \rho}\hat{e}_{\rho}+\dfrac{1}{\rho}\dfrac{\partial}{\partial \phi}\hat{e}_{\phi}+\dfrac{\partial}{\partial z}\hat{e}_z
      $
    }

    \pagebreak

    \item Using the general formula for the divergence in generalized curvilinear coordinates, derive the expression for the divergence in spherical polar coordinates (shown in table 10.3 of the textbook).

      \textcolor{hwColor}{
        $
          \begin{cases}
            x=r ~ sin(\theta) ~ cos(\phi) \\
            y=r ~ sin(\theta) ~ sin(\phi) \\
            z=r ~ cos(\theta)
          \end{cases} \Longleftrightarrow \begin{cases}
            r= \sqrt{x^2+y^2+z^2} \\
            tan(\phi)=\dfrac{y}{x} \\
            tan(\theta)=\dfrac{\sqrt{x^2+y^2}}{z}
          \end{cases} \\
          \\
          \\
          \overrightarrow{\textbf{r}}=x \hat{i}+y \hat{j}+z \hat{k}=r ~ sin(\theta) ~ cos(\phi) \hat{i}+r ~ sin(\theta) ~ sin(\phi) \hat{j}+r ~ cos(\theta) \hat{k} \\
        $
      }

      \textcolor{hwColor}{
        Let's find the derivative of $\textbf{r}$ with respect to $r, ~ \theta, ~ and ~ \phi$ respectively and divide each of the resulting vectors by
        its modulus then we obtain the unit basis vectors. \\
        \\
        $
          \begin{cases}
            \hat{e}_r=sin(\theta) ~ cos(\phi) \hat{i}+sin(\theta) ~ sin(\phi) \hat{j}+cos(\theta) \hat{k} \\
            \\
            \hat{e}_{\theta}=cos(\theta) ~ cos(\phi) \hat{i}+cos(\theta) ~ sin(\phi) \hat{j}-sin(\theta) \hat{k} \\
            \\
            \hat{e}_{\phi}=-sin(\phi) \hat{i}+cos(\phi) \hat{j}
          \end{cases} \\
          \\
          \\
          \overrightarrow{d\textbf{r}}=dr ~ \hat{e}_r+r ~ d\theta ~ \hat{e}_{\theta}+r sin(\theta) ~ d\phi ~ \hat{e}_{\phi} \\
          \\
          \\
          \overrightarrow{\nabla}=\hat{e}_r \dfrac{\partial}{\partial r}+\hat{e}_{\theta} \dfrac{1}{r} \dfrac{\partial}{\partial \theta}+\hat{e}_{\phi} \dfrac{1}{r ~ sin(\theta)} \dfrac{\partial}{\partial \phi} \\
          \\ \hat{e}_{\theta}
          \\
          \overrightarrow{\nabla}.\overrightarrow{a}=\left(\dfrac{\partial}{\partial r} ~ \hat{e}_r+\dfrac{1}{r} \dfrac{\partial}{\partial \theta} ~\hat{e}_{\theta}+\dfrac{1}{r ~ sin(\theta)} \dfrac{\partial}{\partial \phi} ~\hat{e}_{\phi}\right).\overrightarrow{a}=\hat{e}_r ~\dfrac{\partial \overrightarrow{a}}{\partial r}+\hat{e}_{\theta} ~ \dfrac{1}{r} \dfrac{\partial \overrightarrow{a}}{\partial \theta}+\hat{e}_{\phi} ~\dfrac{1}{r sin(\theta)} \dfrac{\partial \overrightarrow{a}}{\partial \phi} \\
          \\
          \\
          =\hat{e}_r\left[\dfrac{\partial a_r}{\partial r} \hat{e}_r+\dfrac{\partial a_{\theta}}{\partial r} \hat{e}_{\theta}+\dfrac{\partial a_{\phi}}{\partial r} \hat{e}_{\phi}+a_r \dfrac{\partial \hat{e}_r}{\partial r}+a_{\theta} \dfrac{\partial \hat{e}_{\theta}}{\partial r}+a_{\phi} \dfrac{\partial \hat{e}_{\phi}}{\partial r}\right] \\ \\
          +\dfrac{\hat{e}_{\theta}}{r}\left[\dfrac{\partial a_r}{\partial \theta} \hat{e}_r+\dfrac{\partial a_{\theta}}{\partial \theta} \hat{e}_{\theta}+\dfrac{\partial a_{\phi}}{\partial \theta} \hat{e}_{\phi}+a_r \dfrac{\partial \hat{e}_r}{\partial \theta}+a_{\theta} \dfrac{\partial \hat{e}_{\theta}}{\partial \theta}+a_{\phi} \dfrac{\partial \hat{e}_{\phi}}{\partial \theta}\right] \\ \\ 
          +\hat{e}_{\phi} \dfrac{1}{r ~ sin(\theta)}\left[\dfrac{\partial a_r}{\partial \phi} \hat{e}_r+\dfrac{\partial a_{\theta}}{\partial \phi} \hat{e}_{\theta}+\dfrac{\partial a_{\phi}}{\partial \phi} \hat{e}_{\phi}+a_r \dfrac{\partial \hat{e}_r}{\partial \phi}+a_{\theta} \dfrac{\partial \hat{e}_{\theta}}{\partial \phi}+a_{\phi} \dfrac{\partial \hat{e}_{\phi}}{\partial \phi}\right] \\
          \\
          \\
          \\
          \overrightarrow{\nabla}.\overrightarrow{a}=\hat{e}_r\left[\dfrac{\partial a_r}{\partial r} \hat{e}_r+\dfrac{\partial a_{\theta}}{\partial r} \hat{e}_{\theta}+\dfrac{\partial a_{\phi}}{\partial r} \hat{e}_{\phi}\right]+\dfrac{\hat{e}_{\theta}}{r}\left[\dfrac{\partial a_r}{\partial \theta} \hat{e}_r+\dfrac{\partial a_{\theta}}{\partial \theta} \hat{e}_{\theta}+\dfrac{\partial a_{\phi}}{\partial \theta} \hat{e}_{\phi}+ a_r \hat{e}_{\theta}+a_{\theta}(-\hat{e}_r)\right]   \\ \\ 
          +\hat{e}_{\phi} \dfrac{1}{r ~ sin(\theta)} \left[\dfrac{\partial a_r}{\partial \phi} \hat{e}_r+\dfrac{\partial a_{\theta}}{\partial \phi} \hat{e}_{\theta}+\dfrac{\partial a_{\phi}}{\partial \phi} \hat{e}_{\phi}+a_r sin(\theta) \hat{e}_{\phi}+a_{\theta} cos(\theta) \hat{e}_{\theta}-a_{\phi}(\hat{e}_r sin(\theta)+\hat{e}_{\theta} cos(\theta))\right] \\ 
          \\
          \\
          \\
          =\dfrac{\partial a_r}{\partial r}+\left[\dfrac{1}{r sin(\theta)}\dfrac{\partial a_{\phi}}{\partial \phi}+\dfrac{a_r}{r}+\dfrac{a_r cos(\theta)}{r sin(\theta)}\right]+\left[\dfrac{1}{r} \dfrac{\partial a_{\theta}}{\partial \theta}+\dfrac{a_r}{r}\right] \\
          \\
          \\
          =\dfrac{1}{r sin(\theta)}\dfrac{\partial a_{\phi}}{\partial \phi}+\left[\dfrac{1}{r} \dfrac{\partial a_{\theta}}{\partial \theta}+\dfrac{a_{\theta} cos(\theta)}{r sin(\theta)}\right]+\left[\dfrac{\partial a_r}{\partial r}+\dfrac{2 a_r}{r}\right] \\
          \\
          \\
          \\
          \Longrightarrow \overrightarrow{\nabla}.\overrightarrow{a}=\dfrac{1}{r^2}\dfrac{\partial}{\partial r}r^2 ~ a_r+\dfrac{1}{r ~ sin(\theta)} \dfrac{\partial }{\partial \theta}a_{\theta} ~ sin(\theta)+\dfrac{1}{r ~ sin(\theta)} \dfrac{\partial a_{\phi}}{\partial \phi}
        $
      }

    \item Using the general formula for the Laplacian of a scalar field in generalized curvilinear coordinates, derive the expression for the Laplacian in spherical polar coordinates (shown in table 10.3 of the textbook). It is recommended that you memorize the result. 

      \textcolor{hwColor}{
        $
          \begin{cases}
            x=r ~ sin(\theta) ~ cos(\phi) \\
            y=r ~ sin(\theta) ~ sin(\phi) \\
            z=r ~ cos(\theta)
          \end{cases} \Longleftrightarrow \begin{cases}
            r= \sqrt{x^2+y^2+z^2} \\
            tan(\phi)=\dfrac{y}{x} \\
            tan(\theta)=\dfrac{\sqrt{x^2+y^2}}{z}
          \end{cases} \\
          \\
          \\
          \overrightarrow{\textbf{r}}=x \hat{i}+y \hat{j}+z \hat{k}=r ~ sin(\theta) ~ cos(\phi) \hat{i}+r ~ sin(\theta) ~ sin(\phi) \hat{j}+r ~ cos(\theta) \hat{k} \\
        $
      }

      \textcolor{hwColor}{
        $
          \begin{cases}
            e_r=\dfrac{\partial \overrightarrow{r}}{\partial r}=sin(\theta) ~ cos(\phi) \hat{i}+sin(\theta) ~ sin(\phi) \hat{j}+cos(\theta) \hat{k} \\
            \\
            e_{\theta}=r ~ cos(\theta) ~ cos(\phi) \hat{i}+r ~ cos(\theta) ~ sin(\phi) \hat{j}-r ~ sin(\theta) \hat{k} \\
            \\
            e_{\phi}=\dfrac{\partial \overrightarrow{r}}{\partial \phi}=-r ~ sin(\theta) ~ sin(\phi) \hat{i}+r ~ sin(\theta) ~ cos(\phi) \hat{j}
          \end{cases} \Longrightarrow \begin{cases}
            \hat{e}_r=e_r \\
            \hat{e}_{\theta}=\dfrac{1}{r} ~ e_{\theta} \\
            \hat{e}_{\phi}=\dfrac{1}{r ~ sin(\theta)} ~ e_{\phi}
          \end{cases}
          \\
          \\
          \begin{cases}
            \hat{e}_r.\overrightarrow{\nabla}=\dfrac{\partial \overrightarrow{r}}{\partial r}.\overrightarrow{\nabla}=\left[\dfrac{\partial x}{\partial r}\dfrac{\partial}{\partial x}+\dfrac{\partial y}{\partial r}\dfrac{\partial}{\partial y}+\dfrac{\partial z}{\partial r}\dfrac{\partial}{\partial z}\right]=\dfrac{\partial}{\partial r} \\
            \\
            \hat{e}_{\theta}.\overrightarrow{\nabla}=\dfrac{1}{r} \dfrac{\partial \overrightarrow{r}}{\partial r}.\overrightarrow{\nabla}=\dfrac{1}{r}\left[\dfrac{\partial x}{\partial r}\dfrac{\partial}{\partial x}+\dfrac{\partial y}{\partial r}\dfrac{\partial}{\partial y}+\dfrac{\partial z}{\partial r}\dfrac{\partial}{\partial z}\right]=\dfrac{1}{r}\dfrac{\partial}{\partial r} \\
            \\
            \hat{e}_{\phi}.\overrightarrow{\nabla}=\dfrac{1}{r ~ sin(\theta)}.\overrightarrow{\nabla}=\dfrac{1}{r ~ sin(\theta)}\dfrac{\partial}{\partial \phi} \\
          \end{cases}
          \\
          \\
          \\
          \overrightarrow{\nabla}=\dfrac{\partial}{\partial r}\hat{e}_r+\dfrac{1}{r}\dfrac{\partial}{\partial \theta}\hat{e}_{\theta}+\dfrac{1}{r ~ sin(\theta)}\dfrac{\partial}{\partial \phi}\hat{e}_{\phi} \\
          \\
          \\
          \\
          \overrightarrow{\nabla}.\overrightarrow{\nabla}=\left[\dfrac{\partial}{\partial r}\hat{e}_r+\dfrac{1}{r}\dfrac{\partial}{\partial \theta}\hat{e}_{\theta}+\dfrac{1}{r ~ sin(\theta)}\dfrac{\partial}{\partial \phi}\hat{e}_{\phi}\right].\left[\dfrac{\partial}{\partial r}\hat{e}_r+\dfrac{1}{r}\dfrac{\partial}{\partial \theta}\hat{e}_{\theta}+\dfrac{1}{r ~ sin(\theta)}\dfrac{\partial}{\partial \phi}\hat{e}_{\phi}\right] \\
          \\
        $
        By doing some partial derivative and alegbra, we have the following results: \\
        \\
        $
          \begin{cases}
            \dfrac{\partial}{\partial r} \hat{e}_r=0, ~~~~~ \dfrac{\partial}{\partial r} \hat{e}_{\theta}=0, ~~~~~ \dfrac{\partial}{\partial r}\hat{e}_{\phi}=0 \\
            \\
            \dfrac{\partial}{\partial \theta} \hat{e}_r=\hat{e}_{\theta}, ~~~~~ \dfrac{\partial}{\partial \theta} \hat{e}_{\theta}=-\hat{e}_{r}, ~~~~~ \dfrac{\partial}{\partial \theta}\hat{e}_{\phi}=0 \\
            \\
            \dfrac{\partial}{\partial \phi} \hat{e}_r=sin(\theta) \hat{e}_{\phi}, ~~~~~ \dfrac{\partial}{\partial \phi} \hat{e}_{\theta}=cos(\theta)\hat{e}_{\phi}, ~~~~~ \dfrac{\partial}{\partial \phi}\hat{e}_{\phi}=-(sin(\theta)\hat{e}_r+cos(\theta) \hat{e}_{\theta}) \\
          \end{cases} \\
          \\
          \\
          \\
          \overrightarrow{\nabla}^2.\Phi=\dfrac{\partial^2}{\partial r^2} \Phi+\dfrac{1}{r}\dfrac{\partial}{\partial r} \Phi+\dfrac{1}{r^2}\dfrac{\partial^2}{\partial \theta^2} \Phi+\dfrac{sin(\theta)}{r ~ sin(\theta)}\dfrac{\partial }{\partial r}\Phi+\dfrac{1}{r^2}\dfrac{cos(\theta)}{sin(\theta)}\dfrac{\partial}{\partial \theta}\Phi+\dfrac{1}{r^2 ~ sin^2(\theta)}\dfrac{\partial^2}{\partial \phi^2}\phi \\
          \\
          \\
          \\
          \Longrightarrow \overrightarrow{\nabla}^2.\Phi=\dfrac{1}{r}\dfrac{\partial^2(r ~ \Phi)}{\partial r^2}+\dfrac{1}{r^2 ~ sin(\theta)}\dfrac{d}{d\theta}(sin(\theta)\dfrac{d\Phi}{d\theta})+\dfrac{1}{r^2 ~ sin^2(\theta)}\dfrac{\partial^2 \Phi}{\partial \phi^2}
        $
      }

    \item Refer to the notion of metric discussed in Theory segment 9.  Generalize the expression of the metric matrix for the case where the basis vectors of the curvilinear coordinate system are not mutually orthogonal.  [Hint: take the expression for  $d\mathbf{r}$, eq. (10.57) of the textbook, and calculate the expression of $d\mathbf{r} \cdot d\mathbf{r}$ for the most general case.]

      \textcolor{hwColor}{
        Let's define a set of 3 new variables that describe the position of a point in space.
        Then, one can write equations that relate the cartesian coordinates to the new coordinates and vice versa\\
        \\
        $
          \begin{cases}
            x=x(u_1, u_2, u_3), ~~~~~~~~~~ u_1=u_1(x,y,z) \\
            y=y(u_1, u_2, u_3), ~~~~~~~~~~ u_2=u_2(x,y,z) \\
            z=z(u_1, u_2, u_3), ~~~~~~~~~~ u_3=u_3(x,y,z) \\
          \end{cases}
        $ \\
        \\
        The new vectors are chosen to be the derivatives of the position vector, properly normalized to unity: \\
        \\
        $
          d\overrightarrow{r}=\dfrac{\partial \overrightarrow{r}}{\partial u_1}du_1+\dfrac{\partial \overrightarrow{r}}{\partial u_2}du_2+\dfrac{\partial \overrightarrow{r}}{\partial u_3}du_3 \\
          \\
          \\
          \begin{cases}
            \hat{e}_1=\dfrac{1}{h_1}\dfrac{\partial \overrightarrow{r}}{\partial u_1}
            \\
            \\
            \hat{e}_2=\dfrac{1}{h_2}\dfrac{\partial \overrightarrow{r}}{\partial u_2}
            \\
            \\
            \hat{e}_3=\dfrac{1}{h_3}\dfrac{\partial \overrightarrow{r}}{\partial u_3} 
          \end{cases} ~~~~~~~~~~~~~~~~ \begin{cases}
            h_1=|\dfrac{\partial \overrightarrow{r}}{\partial u_1}|
            \\
            \\
            h_2=|\dfrac{\partial \overrightarrow{r}}{\partial u_2}|
            \\
            \\
            h_3=|\dfrac{\partial \overrightarrow{r}}{\partial u_3}|
          \end{cases}
        $
        \\
        \\
        \\
        Now we can write $dr$ in terms of the new basis vectors. $d\overrightarrow{r}=h_1 ~ du_1 ~ \hat{e}_1+h_2 ~ du_2 ~ \hat{e}_2+h_3 ~ du_3 ~ \hat{e}_3$ \\
        \\
        $
          d\overrightarrow{r}.d\overrightarrow{r}=\left(h_1 ~ du_1 ~ \hat{e}_1+h_2 ~ du_2 ~ \hat{e}_2+h_3 ~ du_3 ~ \hat{e}_3\right).\left(h_1 ~ du_1 ~ \hat{e}_1+h_2 ~ du_2 ~ \hat{e}_2+h_3 ~ du_3 ~ \hat{e}_3\right) \\
          \\
          =h_1 ~ du_1 ~ \hat{e}_1 h_1 ~ du_1 ~ \hat{e}_1+h_1 ~ du_1 ~ \hat{e}_1 h_2 ~ du_2 ~ \hat{e}_2+h_1 ~ du_1 ~ \hat{e}_1 h_3 ~ du_3 ~ \hat{e}_3 \\
          \\
          +h_2 ~ du_2 ~ \hat{e}_2 h_1 ~ du_1 ~ \hat{e}_1+h_2 ~ du_2 ~ \hat{e}_2 h_2 ~ du_2 ~ \hat{e}_2+h_2 ~ du_2 ~ \hat{e}_2 h_3 ~ du_3 ~ \hat{e}_3 \\
          \\
          +h_3 ~ du_3 ~ \hat{e}_3 h_1 ~ du_1 ~ \hat{e}_1+h_3 ~ du_3 ~ \hat{e}_3 h_2 ~ du_2 ~ \hat{e}_2+h_3 ~ du_3 ~ \hat{e}_3 h_3 ~ du_3 ~ \hat{e}_3 \\
          \\
          \\
          d\overrightarrow{r}.d\overrightarrow{r}=\begin{pmatrix}
            du_1 & du_2 & du_3
          \end{pmatrix} \begin{pmatrix}
            h^2_1 \hat{e}_1 \hat{e}_1 & h_1 h_2 \dfrac{1}{du_1} du_2 \hat{e}_1 \hat{e}_2 & h_1 h_3 \dfrac{1}{du_1} du_3 \hat{e}_1 \hat{e}_3 \\
            \\
            h_1 h_2 du_1 \dfrac{1}{du_2} \hat{e}_1 \hat{e}_2 & h^2_2 \hat{e}_2 \hat{e}_2 & h_2 h_3 \dfrac{1}{du_2} \hat{e}_2 \hat{e}_3 \\
            \\
            h_1 h_3 du_1 \dfrac{1}{du_3} \hat{e}_1 \hat{e}_3 & h_2 h_3 du_2 \dfrac{1}{du_3} \hat{e}_2 \hat{e}_3& h^2_3 \hat{e}_3 \hat{e}_3 \\
          \end{pmatrix} \begin{pmatrix}
            du_1 \\
            du_2 \\
            du_3
          \end{pmatrix}
        $
        \\
        \\
        The metric is not diagonal and the three components of $dr$ along the unit vectors are not mutually orthogonal, otherwise the metric matrix
        would have zeros everywhere except the main diagonal. 
      }

    \item Derive the expression of the metric for cartesian coordinates. Comment on the results as adequate.
    
    \textcolor{hwColor}{
      $
        \overrightarrow{r}=x \hat{i}+y\hat{j}+z\hat{k} \\
        \\
        d\overrightarrow{r}=\dfrac{\partial \overrightarrow{r}}{\partial x} dx+\dfrac{\partial \overrightarrow{r}}{\partial y} dy+\dfrac{\partial \overrightarrow{r}}{\partial z}dz \\
        \\
        \begin{cases}
          \hat{e}_x=\dfrac{\dfrac{\partial \overrightarrow{r}}{\partial x}}{|\dfrac{\partial \overrightarrow{r}}{\partial x}|} \\
          \\
          \hat{e}_y=\dfrac{\dfrac{\partial \overrightarrow{r}}{\partial y}}{|\dfrac{\partial \overrightarrow{r}}{\partial y}|}  \\
          \\
          \hat{e}_z=\dfrac{\dfrac{\partial \overrightarrow{r}}{\partial z}}{|\dfrac{\partial \overrightarrow{r}}{\partial z}|}  \\
        \end{cases} \\
        \\
      $
      \\
      \\
      \\
      A generic vector can be written in the new basis as: \\
      \\
      $\overrightarrow{r}=\alpha_1 \hat{e}_x+\alpha_2 \hat{e}_y+\alpha_3 \hat{e}_z$ where $\alpha_1=\overrightarrow{r}.\hat{e}_x, ~~ \alpha_2=\overrightarrow{r}.\hat{e}_y,$ and 
      $\alpha_3=\overrightarrow{r}.\hat{e}_z$
      \\
      \\
      \\
      $
        d\overrightarrow{r}.d\overrightarrow{r}=\left[x \hat{e}_x+y \hat{e}_y+z \hat{e}_z\right].\left[x \hat{e}_x+y \hat{e}_y+z \hat{e}_z\right] \\
        \\
        =xx\hat{e}_x\hat{e}_x+xy \hat{e}_x \hat{e}_y+xz \hat{e}_x \hat{e}_z+yx \hat{e}_x \hat{e}_y+yy\hat{e}_y \hat{e}_y+yz \hat{e}_y \hat{e}_z+zx \hat{e}_z \hat{e}_x+zy \hat{e}_y \hat{e}_z+zz \hat{e}_z \hat{e}_z
      $
      \\
      For cartesian coordinate, we know that the basis vectors are orthogonal, hence: \\
      \\
      $
        \begin{cases}
          \hat{e}_x\hat{e}_y=\hat{e}_x\hat{e}_z=\hat{e}_y\hat{e}_z=0 \\
          \\
          \hat{e}_x\hat{e}_x=\hat{e}_y\hat{e}_y=\hat{e}_z\hat{e}_z=1
        \end{cases} \\
        \\
        \\
        d\overrightarrow{r}.d\overrightarrow{r}=xx\hat{e}_x+yy\hat{e}_y+zz\hat{e}_z \\
        \\
        \\
        \begin{pmatrix}
          xx & 0 & 0 \\
          0 & yy & 0 \\
          0 & 0 & zz
        \end{pmatrix} \begin{pmatrix}
          \hat{e}_x \\
          \hat{e}_y \\
          \hat{e}_z
        \end{pmatrix}
      $ \\
      \\
      The metric is diagonal and the three components of $dr$ along the unit vectors are mutually orthogonal.
    }

    
  \end{enumerate}


  \textbf{Part B}
  \begin{enumerate}
    \item  Perform the scalar field path integral, $\bigints_A^B xy ds$, where $A=(0,0)$ and $B=(1,1/2)$ over each of the following three paths:
      \begin{enumerate}
      \item Along the x-axis to $(1, 0)$ and then up to $B$

        \textcolor{hwColor}{
          $
          \bigints_A^B f(x,y) ds=\bigints_{t=a}^{t=b} f(x(t), y(t)) ~ |\dfrac{d\overrightarrow{r}}{dt}| dt \\ \\
            x=c_1+k_1, ~~ y=c_2+k_2t \\\
            \
            \begin{cases}
              x(t=0): ~ 1=c_1+0 \Rightarrow c_1=1 ~~ and ~~ x(t=1): ~ 1=1+k_1(1) \Longrightarrow x=1 \\
              \\
              y(t=0): ~ 0=c_2+0 \Rightarrow c_2=0 ~~ and ~~ y(t=1): ~ \dfrac{1}{2}=0+k_2(1) \Longrightarrow y=\dfrac{1}{2}t
            \end{cases} \\
            \\
            \\
            \overrightarrow{r}(t)=\hat{i}+\dfrac{1}{2}t \hat{j} \Rightarrow \dfrac{d\overrightarrow{r}}{dt}=\dfrac{1}{2} \hat{j} \Longrightarrow |\dfrac{d\overrightarrow{r}}{dt}|=\dfrac{1}{2} \\
            \\
            \\
            \bigints_{t=0}^{t=1} (1)(\dfrac{1}{2}t) ~ (\dfrac{1}{2}) dt=\dfrac{1}{8}
          $
        }

      \pagebreak

      \item Up the y-axis to $(0, 1/2)$ and then over to $B$

        \textcolor{hwColor}{
          $
          \bigints_A^B f(x,y) ds=\bigints_{t=a}^{t=b} f(x(t), y(t)) ~ |\dfrac{d\overrightarrow{r}}{dt}| dt \\ \\
            x=c_1+k_1, ~~ y=c_2+k_2t \\
            \
            \begin{cases}
              x(t=0): ~ 0=c_1+0 \Rightarrow c_1=0 ~~ and ~~ x(t=1): ~ 1=k_1(1) \Longrightarrow x=t \\
              \\
              y(t=0): ~ \dfrac{1}{2}=c_2+0 \Rightarrow c_2=\dfrac{1}{2} ~~ and ~~ y(t=1): ~ \dfrac{1}{2}=\dfrac{1}{2}+k_2(1) \Longrightarrow y=\dfrac{1}{2}
            \end{cases} \\
            \\
            \\
            \overrightarrow{r}(t)=t\hat{i}+\dfrac{1}{2} \hat{j} \Rightarrow \dfrac{d\overrightarrow{r}}{dt}=\hat{i} \Longrightarrow |\dfrac{d\overrightarrow{r}}{dt}|=1 \\
            \\
            \\
            \bigints_{t=0}^{t=1} (t)(\dfrac{1}{2}) ~ (1) dt=\dfrac{1}{4} \\
          $
        }

      \item Along the curve $y = x^2/2$

        \textcolor{hwColor}{
          For this case, we can do trivial parametric equation meaning we call the dependent variable t in this case x. Hence, \\ \\ 
          $
            x=t \Rightarrow y=\dfrac{t^2}{2} \\
            \\
          $
          Since we use the above method then $0\le t \le 1$ \\
          \\
          $
            \overrightarrow{r}(t)=t\hat{i}+\dfrac{t^2}{2} \hat{j} \Rightarrow \dfrac{d\overrightarrow{r}}{dt}=\hat{i}+t\hat{j} \Longrightarrow |\dfrac{d\overrightarrow{r}}{dt}|=\sqrt{1+t^2}
            \\
            \\
            \bigints_A^B f(x,y) ds=\bigints_{t=a}^{t=b} f(x(t), y(t)) ~ |\dfrac{d\overrightarrow{r}}{dt}| dt=\bigints_{t=0}^{t=1} (t)(\dfrac{t^2}{2}) ~ \sqrt{1+t^2} dt \\
            \\
            \\
            =\dfrac{1}{2}\bigints_{t=0}^{t=1} t^3 ~ \sqrt{1+t^2} dt=\dfrac{\sqrt{2}+1}{15}
          $
        }

      \end{enumerate}

    \item Compute the vector path integral $\bigints_A^B \mathbf{V}\cdot d\mathbf{s}$, where $\mathbf{V}=y \mathbf{\hat i} + x \mathbf{\hat j}$, $A=(0,0)$, and $B=(1,1)$. Do the calculation for two different paths of your choice.

    (Hint:  The vector $ds$ is $ds=\hat{i}dx+\hat{j}dy=dx(\hat{i}+\dfrac{dy}{dx} \hat{j})$.  
    You should obtain ordinary integrals over $x$ or $y$.  )

    \item Do the same integrals as in exercise above, for $\mathbf{V}=y \mathbf{\hat i} - x \mathbf{\hat j}$. 

  \end{enumerate}

\end{document}
