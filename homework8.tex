\documentclass[fleqn]{article}
\oddsidemargin 0.0in
\textwidth 6.0in
\thispagestyle{empty}
\usepackage{import}
\usepackage{amsmath}
\usepackage{graphicx}
\usepackage{flexisym}
\usepackage{amssymb}
\usepackage{bigints} 
\usepackage[english]{babel}
\usepackage[utf8x]{inputenc}
\usepackage{float}
\usepackage[colorinlistoftodos]{todonotes}

\definecolor{hwColor}{HTML}{AD53BA}

\begin{document}

  \begin{titlepage}

    \newcommand{\HRule}{\rule{\linewidth}{0.5mm}}

    \center



    \textsc{\LARGE Arizona State University}\\[1.5cm]

    \textsc{\LARGE Mathematical Methods For Physics II }\\[1.5cm] 


    \begin{figure}
      \includegraphics[width=\linewidth]{asu.png}
    \end{figure}


    \HRule \\[0.4cm]
    { \huge \bfseries Homework Eight}\\[0.4cm] 
    \HRule \\[1.5cm]

    \textbf{Behnam Amiri}

    \bigbreak

    \textbf{Prof: Cecilia Lunardini}

    \bigbreak


    \textbf{{\large \today}\\[2cm]}

    \vfill

  \end{titlepage}

  \textbf{Part A}
  \begin{enumerate}
    \item Fill the missing steps in sec. 18.5 of the textbook. Starting from eq. (18.74) go as far as proving eqs. (18.77) and (18.78).  


    \item Continue filling in the steps starting with eq. (18.78) and go as far as proving eq. (18.79) (upper line only). 
    
    
    \item Prove the result at the bottom of the page 606 of the textbook, i.e., that for $\nu$ integer, $J_{-\nu }(x)=(-1)^\nu J_{\nu }(x)$. \\
    (Hint: Start with the series expression for $J_{-\nu }(x)$, which is given by Eq. 18.79 of the textbook where you have to replace $\nu \rightarrow -\nu$. Note that if $k$ is a non-negative integer, then $(-1)^k \Gamma(-k) = \infty$, see fig. 18.9 of the textbook. Eliminate those terms from the sum that equal zero because of this, and then change the summation variable.)
    
  \end{enumerate}


  \textbf{Part B}
  \begin{enumerate}
    \item For a system in spherical polar coordinates, the radial part of Laplace's equation is
    $$r^2 \dfrac{d^2 R}{dr^2}+2r \dfrac{dR}{dr}+\left[k^2r^2-n(n+1)\right]R=0$$
    Using the method shown in class (class 16, example segment on transforming a second order differential equation into Bessel's equation), find it solution in terms of two linearly independent Bessel functions. (Note: write your solution in terms of one Bessel function of the first kind and one Bessel function of the second kind). 


    \item Using the series expression of the Bessel function $J_{\nu}$, prove eq. (18.93) of the textbook. 


    \item Using a graphing tool of your choice (e.g., Wolfram alpha, where all you need is a browser), plot $J_4$ and $J_5$, and verify that the $n-$th root of $J_5$ is always larger than the $n-$th root of $J_4$ (for the same $n$; do not count $x=0$ as a root).  In other words, that $\zeta_{4n}<\zeta_{5n}$ for all $n$. (include the graph in your submission; it will suffice to check the result for $n=1,2,3,4$). 


  \end{enumerate}

\end{document}
