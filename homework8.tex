\documentclass[fleqn]{article}
\oddsidemargin 0.0in
\textwidth 6.0in
\thispagestyle{empty}
\usepackage{import}
\usepackage{amsmath}
\usepackage{graphicx}
\usepackage{flexisym}
\usepackage{amssymb}
\usepackage{bigints} 
\usepackage[english]{babel}
\usepackage[utf8x]{inputenc}
\usepackage{float}
\usepackage[colorinlistoftodos]{todonotes}

\definecolor{hwColor}{HTML}{AD53BA}

\begin{document}

  \begin{titlepage}

    \newcommand{\HRule}{\rule{\linewidth}{0.5mm}}

    \center



    \textsc{\LARGE Arizona State University}\\[1.5cm]

    \textsc{\LARGE Mathematical Methods For Physics II }\\[1.5cm] 


    \begin{figure}
      \includegraphics[width=\linewidth]{asu.png}
    \end{figure}


    \HRule \\[0.4cm]
    { \huge \bfseries Homework Eight}\\[0.4cm] 
    \HRule \\[1.5cm]

    \textbf{Behnam Amiri}

    \bigbreak

    \textbf{Prof: Cecilia Lunardini}

    \bigbreak


    \textbf{{\large \today}\\[2cm]}

    \vfill

  \end{titlepage}

  \textbf{Part A}
  \begin{enumerate}
    \item Fill the missing steps in sec. 18.5 of the textbook. Starting from eq. (18.74) go as far as proving eqs. (18.77) and (18.78). 
    
      \textcolor{hwColor}{
        Let's start off with the Bessel's differential equation. (first kind) \\
        \\
        $
          \dfrac{d^2 y}{dx^2}+\dfrac{1}{x} \dfrac{dy}{dx}+\left(1-\dfrac{v^2}{x^2}\right)y=0
        $ 
        \\
        \\
        From the textbook (page 603) "By inspection, $x=0$ is a regular sigular point;...", hence, \\
        \\
        \\
        $
          y=x^{\sigma} \sum\limits_{n=0}^{\infty} a_n x^n=\sum\limits_{n=0}^{\infty} a_nx^{n+\sigma} \\
          \\
          \dfrac{dy}{dx}=\dfrac{d}{dx}\left[\sum\limits_{n=0}^{\infty} a_nx^{n+\sigma}\right]=\sum\limits_{n=0}^{\infty} a_n (n+\sigma) x^{n+\sigma-1} \\
          \\
          \\
          \\
          \dfrac{d^2y}{dx^2}=\dfrac{d}{dx}\left[\sum\limits_{n=0}^{\infty} a_n (n+\sigma) x^{n+\sigma-1}\right]=\sum\limits_{n=0}^{\infty} a_n (n^2+\sigma^2+2n\sigma-n-\sigma) x^{n+\sigma-2} \\
          \\
        $ 
        Now it's time to plugin what we found into the first kind Bessel's differential equation. \\
        \\
        $
          \dfrac{d^2 y}{dx^2}+\dfrac{1}{x} \dfrac{dy}{dx}+\left(1-\dfrac{v^2}{x^2}\right)y=0 \\
          \\
          \\
          \left[\sum\limits_{n=0}^{\infty} a_n (n^2+\sigma^2+2n\sigma-n-\sigma) x^{n+\sigma-2}\right]
          +\dfrac{1}{x} \left[\sum\limits_{n=0}^{\infty} a_n (n+\sigma) x^{n+\sigma-1}\right] \\
          +\left(1-\dfrac{v^2}{x^2}\right) \left[\sum\limits_{n=0}^{\infty} a_nx^{n+\sigma} \right]=0 \\
          \\
          \\
          \sum\limits_{n=0}^{\infty} \left[(n+\sigma)-v^2+(n+\sigma-1)(n+\sigma)\right] a_n x^{n+\sigma-2}+\sum\limits_{n=0}^{\infty} a_n x^{n+\sigma}=0 \\
          \\
          \\
          \left[\sum\limits_{n=0}^{\infty} \left[(n+\sigma)-v^2+(n+\sigma-1)(n+\sigma)\right] a_n x^{n+\sigma-2}+\sum\limits_{n=0}^{\infty} a_n x^{n+\sigma}\right] \times x^{2-\sigma}=0 \times x^{2-\sigma} \\
          \\
          \\
          \sum\limits_{n=0}^{\infty}a_n x^{n+2}+\sum\limits_{n=0}^{\infty} \left[(n+\sigma)^2-v^2\right]a_n x^n=0
          \\
          \\
        $
        For the coefficient of x when $n=0$ and $n=1$ we have: \textbf{(The Frobenius solution)} \\ \\
        \emph{Side note: a good way to think about it is that if $a_0$ was 0, the recursion relationship 
        would say that all terms are zero because $a_0$ will be subbed into every other term(the evens at least)
        since we get a contradiction it implies that $a_0 \neq 0 $ so the other portion must} \\ \\
        $
          \begin{cases}
            n=0 ~ \rightarrow ~ \left[(0+\sigma)^2-v^2\right]a_0=\left[\sigma^2-v^2\right]a_0=0 \Rightarrow \sigma=\pm v \\
            \\
            n=1 ~ \rightarrow ~ \left[(1+\sigma)^2-v^2\right]a_1 \Rightarrow \left[(1+\sigma)^2-v^2\right]a_1b \\
          \end{cases} \\
          \\
        $
        For power of 2 and higher, we have this coefficient: \\
        \\
        $
          \left[(n+\sigma)^2-v^2\right]a_n+a_{n-2}=0
        $
        \\
        \\
        A for $x^2$ and higher powers; \\
        \\
        $
          \sum\limits_{n=0}^{\infty}a_n \left[(n+\sigma)^2-v^2\right]+\sum\limits_{n=0}^{\infty} a_{n-2} x^n=0 \\
        $
        \\ 
        \\
        For sigma we can find the recurrence relations as,  \\
        \\
        $
          \begin{cases}
            \left[(1+\sigma)^2-v^2\right]a_1=0 \Rightarrow \left[(1\pm v)^2-v^2\right]a_1=\left[1\pm+2v+v^2-v^2\right]a_1=0 \Rightarrow ~ (1\pm 2v)a_1=0 \\
            \\
            \left[(n+\sigma)^2-v^2\right]a_n+a_{n-2}=0 \Rightarrow \left[n^2 \pm 2nv+v^2-v^2\right]a_n+a_{n-2}=0 \\
          \end{cases} \\
          \\ \\
          \therefore ~~ n(n\pm 2v)a_n+a_{n-2}=0 
        $
         for $n \geq 2 \in \mathcal{Z}$
      }

    \item Continue filling in the steps starting with eq. (18.78) and go as far as proving eq. (18.79) (upper line only). 
    
      \textcolor{hwColor}{
        From the textbook (18.78) is $n(n\pm 2v)a_n+a_{n-2}=0$ for $n \geq 2$. From this equation 
        $a_n$ can be found as: 
        \\
        \\
        $
          a_n=-\dfrac{a_{n-2}}{n(n \pm 2v)} \\
          \\
          \\
          \begin{cases}
            a_2=-\dfrac{a_0}{4(1 \pm v)} \\
            \\
            a_4=-\dfrac{a_2}{4(4 \pm 2v)}=\dfrac{a_0}{32(1 \pm)(2 \pm v)} \\
            \\
            a_6=-\dfrac{a_{4}}{6(6 \pm 2v)}=\dfrac{a_0}{384(1 \pm)(2 \pm v)(3 \pm v)}
          \end{cases} \\ \\
        $
        Time to find our solution: \\
        \\
        $
          y=x^{\sigma} \sum\limits_{n=0}^{\infty}a_n x^n \Rightarrow \mathcal{J}_{v}(x)=x^v \left[a_0+a_2+a_4+a_6+...\right] \\
          \\
          \\
          \mathcal{J}_{v}(x)=x^v \left[a_0-\dfrac{a_0 x^2}{4(1 \pm v)}+ \dfrac{a_0 x^4}{32(1 \pm)(2 \pm v)}+ \dfrac{a_0 x^6}{384(1 \pm)(2 \pm v)(3 \pm v)}+...\right] \\
          \\
          \\
          =x^v a_0 \left[-\dfrac{x^2}{4(1 \pm v)}+ \dfrac{x^4}{32(1 \pm)(2 \pm v)}+ \dfrac{x^6}{384(1 \pm)(2 \pm v)(3 \pm v)}+...\right] \\
          \\
          a_0 \equiv \dfrac{1}{2^v \Gamma \left(1+v\right)}, ~
        $
         then: \\
         \\
         $
          \mathcal{J}_{v}(x)=\dfrac{x^v}{2^v \Gamma \left(1+v\right)} \left[-\dfrac{x^2}{4(1 \pm v)}+ \dfrac{x^4}{32(1 \pm)(2 \pm v)}+ \dfrac{x^6}{384(1 \pm)(2 \pm v)(3 \pm v)}+...\right] \\
          \\
          \\
          =\dfrac{1}{\Gamma \left(1+v\right)} \left(\dfrac{x}{2}\right) \left[1-\dfrac{1}{(1+v)}\left(\dfrac{x}{2}\right)^2+\dfrac{1}{2(1+v)(2+v)}\left(\dfrac{x}{2}\right)^4-\dfrac{1}{6 (1+v)(2+v)(3+v)} \left(\dfrac{x}{2}\right)^6+...\right] \\
          \\
          \\
          \therefore ~~ \mathcal{J}_{v}(x)=\sum\limits_{n=0}^{\infty} \dfrac{(-1)^n}{n! \Gamma \left[1+v+n\right]} \left(\dfrac{x}{2}\right)^{2n+v}
         $
      }
    
    \item Prove the result at the bottom of the page 606 of the textbook, i.e., that for $\nu$ integer, $J_{-\nu }(x)=(-1)^\nu J_{\nu }(x)$. \\
    (Hint: Start with the series expression for $J_{-\nu }(x)$, which is given by Eq. 18.79 of the textbook where you have to replace $\nu \rightarrow -\nu$. Note that if $k$ is a non-negative integer, then $(-1)^k \Gamma(-k) = \infty$, see fig. 18.9 of the textbook. Eliminate those terms from the sum that equal zero because of this, and then change the summation variable.)
    
      \textcolor{hwColor}{
        The definition of the Bessel function $\mathcal{J}_{v}(x)$ is valid for all values of $v$. In the case of integer $v$ the general 
        solution of Bessel's equation cannot be written in the form of $y(x)=c_1 \mathcal{J}_{v}(x)+c_2 \mathcal{J}_{-v}(x)$. \\
        \\
        The gamma function is defined for all complex numbers except the non-positive integers. For any positive 
        integer $n$, $\Gamma(n)=\left(n-1\right)!$. The gamma function can be extended to all real and complex values 
        (except the negative integers and zero). Based on the definition of the gamma function
        when the domain is a negative number then the gamma function is undefined. From the lecture, we also learned
        that $\gamma(x+1)=n!$. With all these said, we have: \\ 
        \\
        \\
        $
          \mathcal{J}_{-v}(x)=\sum\limits_{n=0}^{\infty} \dfrac{(-1)^n}{n! \Gamma(1-v+n)} \left(\dfrac{x}{2}\right)^{2n-v} \\
          \\
          =(-1)^v \sum\limits_{n=0}^{\infty} \dfrac{(-1)^n}{\Gamma(v+n+1)\Gamma(1+n)} \left(\dfrac{x}{2}\right)^{2n+v} \\
          \\
          =(-1)^v \sum\limits_{n=0}^{\infty} \dfrac{(-1)^n}{n! \Gamma(1+n+v)} \left(\dfrac{x}{2}\right)^{2n+v} \\
          \\
          \therefore ~~ \mathcal{J}_{-v}(x)=(-1)^v \mathcal{J}_{v}(x)
        $
      }

  \end{enumerate}

  \pagebreak

  \textbf{Part B}
  \begin{enumerate}
    \item For a system in spherical polar coordinates, the radial part of Laplace's equation is
    $$r^2 \dfrac{d^2 R}{dr^2}+2r \dfrac{dR}{dr}+\left[k^2r^2-n(n+1)\right]R=0$$
    Using the method shown in class (class 16, example segment on transforming a second order differential equation into Bessel's equation), find it solution in terms of two linearly independent Bessel functions. (Note: write your solution in terms of one Bessel function of the first kind and one Bessel function of the second kind). 

      \textcolor{hwColor}{
        Let's start off by talking a bit about Bessel functions. They are solutions $y(x)$ of
        Bessel's differential equation. \\
        $$x^2 \dfrac{d^2 y}{dx^2}+x \dfrac{dy}{dx}+\left(x^2-\alpha^2\right)y=0$$ \\
        \\
        $$y(x)=W ~ J_{\alpha}+V ~ Y_{\alpha}$$
        \\
        Even though $\alpha$ and $-\alpha$ produce the same differential equation, it is conventional to 
        define different Bessel functions for these two values in such a way that the Bessel functions 
        are mostly smooth functions of $\alpha$. \\
        \\
        The most important cases are when $\alpha$ is an \textbf{integer} or \textbf{half-integer}. Bessel functions for integer 
        $\alpha$ are also known as cylinder functions or the cylindrical harmonics because they appear in the solution 
        to Laplace's equation in cylindrical coordinates. Spherical Bessel functions with half-integer $\alpha$ are obtained 
        when the Helmholtz equation is solved in spherical coordinates. \\
        \\
        \textbf{Bessel functions of the first kind:} $J_{\alpha}$ \\
        Bessel functions of the first kind, denoted as $J_{\alpha}(x)$, are solutions of Bessel's differential equation. 
        For integer or positive $\alpha$, Bessel functions of the first kind are finite at the origin (x = 0); while for negative 
        non-integer $\alpha$, Bessel functions of the first kind diverge as x approaches zero.
        \\
        \\
        \\
        \textbf{Bessel functions of the second kind:} $Y_{\alpha}$ \\
        The Bessel functions of the second kind, denoted by $Y_{\alpha}(x)$, are solutions of the Bessel differential equation
        that have a singularity at the origin (x = 0) and are multivalued. \\ \\
      }


      \textcolor{hwColor}{
        For a Bessel's equation like this $u^2 \dfrac{d^2 f}{du^2}+u\dfrac{df}{du}+\left(u^2-m^2\right)f=0$, we have: \\
        \\
        $
          u=Kx^{\beta}, ~~~~ y(x)=x^{\alpha} f(u) \\ \\
          \begin{cases}
            \dfrac{dy}{dx}=\dfrac{d}{dx}\left(x^{\alpha f}\right)=\alpha x^{\alpha-1}+x^{\alpha} \dfrac{df}{du} \dfrac{du}{dx} \\
            \\
            \dfrac{du}{dx}=\dfrac{d}{dx}\left(C x^{\beta}\right)=\beta Cx^{\beta -1}
          \end{cases} \longrightarrow
          \dfrac{dy}{dx}=x^{\alpha-1} \left[\alpha f+\beta C x^{\alpha} \dfrac{df}{du}\right] \\
          \\
          \\
          \dfrac{d^2 f}{du^2} \times \dfrac{du}{dx}=\dfrac{d}{dx} \left(\dfrac{df}{du}\right)=\dfrac{d}{du} \left(\dfrac{df}{du}\right) \times \dfrac{du}{dx}
          \Longrightarrow \dfrac{1}{\beta C x^{\beta -1}} \dfrac{d}{dx} \left(\dfrac{df}{du}\right) \\
          \\
          \dfrac{d^2 f}{du^2}=\dfrac{1}{\left(\beta C\right)^2 x^{2 \beta+\alpha}} \left(\alpha(\alpha+\beta)y+(1-2\alpha-\beta)x\dfrac{dy}{dx}+x^2\dfrac{d^2 y}{dx^2}\right) \\
          \\
          \\
          \dfrac{1}{\left(\beta C\right)^2 x^{2 \beta+\alpha}} \left(\alpha(\alpha+\beta)y+(1-2\alpha-\beta)x\dfrac{dy}{dx}+x^2\dfrac{d^2 y}{dx^2}\right)
          +\dfrac{C x^{\beta}}{\beta C x^{\alpha+\beta}} \left[x \dfrac{dy}{dx}-\alpha y\right]
          +\left[C^2 x^{2\beta}-m^2\right]\dfrac{y}{x^{\alpha}}=0 \\ \\
        $
        This reduces to: \\
        \\
        $
          x^2 \dfrac{d^2 y}{dx^2}+\left(1-2 \alpha\right)x\dfrac{dy}{dx}+\left[\alpha^2+\beta^2\left(C^2 x^{2\beta}-m^2\right)\right]y=0
        $ \\
        \\
        This equation gives us three parameters $\left(\alpha, ~~ \beta, ~~ and ~~ C\right)$ that we can use to transform our second-order differential equations.
        the goal is to transform the given differntial equation into a Bessel's equation form. Hence, \\
        \\
        \\
        $
          \begin{cases}
            x^2 \dfrac{d^2 y}{dx^2}+\left(1-2 \alpha\right)x\dfrac{dy}{dx}+\left[\alpha^2+\beta^2\left(C^2 x^{2\beta}-m^2\right)\right]y=0 \\
            \\
            r^2 \dfrac{d^2 R}{dr^2}+2r \dfrac{dR}{dr}+\left[k^2r^2-n(n+1)\right]R=0
          \end{cases} \Rightarrow 1-2 \alpha=2 \Rightarrow \alpha=-\dfrac{1}{2}
        $ \\
        \\ 
        \\
        Now, therefore: \\
        \\
        $
          r^2 \dfrac{d^2 R}{dr^2}+2r \dfrac{dR}{dr}+\left[\dfrac{1}{4}+\beta^2\left(C^2 x^{2\beta}-m^2\right)\right]R=0
        $ \\
        \\
        By equating, $\left[\dfrac{1}{4}+\beta^2\left(C^2 x^{2\beta}-m^2\right)\right]R$ to the given P.D.E, we get: \\
        \\
        $
          \left[\dfrac{1}{4}+\beta^2\left(C^2 x^{2\beta}-m^2\right)\right]R=\left[k^2r^2-n(n+1)\right]R \Longrightarrow 
          \dfrac{1}{4}+\beta^2\left(C^2 x^{2\beta}-m^2\right)=k^2r^2-n(n+1) \\
          \\
          \\
          \therefore ~~ \dfrac{1}{4}+\beta^2 C^2 x^{2\beta}-\beta^2 m^2=k^2r^2-n(n+1) \\
          \\
        $
        Asssuming $\beta=1$ we have $\dfrac{1}{4}+C^2 x^2-m^2=k^2r^2-n(n+1)$.\\ \\
        $
          m^2-\dfrac{1}{4}=n(n+1) \\
          \\
          \therefore ~~ m=\pm \left(n+\dfrac{1}{2}\right) ~~~.
        $ (Note: $k^2 r^2=C^2 x^2$) \\
        \\
        We are almost done since we found all the three parameters: \\
        \\
        $
          \begin{cases}
            \alpha=-\dfrac{1}{2} \\
            \\
            \beta=1 \\
            \\
            C=\pm k \\
            \\
            m=\left(n+\dfrac{1}{2}\right)
          \end{cases}
        $
        \\
        \\
        \\
        As we mentioned earlier the solution for the given differential equation is $ y(x)=x^{\alpha} f(u)$, therefore: \\
        \\
        $
          y(x)=y(x)=x^{\alpha} f(u)=W x^{\alpha} J_m \left(K x^{\beta}\right)+V x^{\alpha} Y_m \left(Kx^{\beta}\right) \\
          \\
          =W x^{-\dfrac{1}{2}} J_{m} \left(K x\right)+V x^{-\dfrac{1}{2}} Y_{m} \left(Kx\right) \\
          \\
          \\
          \\
          \therefore ~~~~ y(x)=W x^{-\dfrac{1}{2}} J_{n+\dfrac{1}{2}} \left(K x\right)+V x^{-\dfrac{1}{2}} Y_{n+\dfrac{1}{2}} \left(Kx\right) \\
          \\
        $
        $W$ and $V$ can be determined by boundry conditions.
      }

    \item Using the series expression of the Bessel function $J_{\nu}$, prove eq. (18.93) of the textbook. 

      \textcolor{hwColor}{
        $
          \dfrac{d}{dx} \left(x^{-v} \mathcal{J}_{v}(x)\right)=-x^{-v} \mathcal{J}_{v+1}(x) \\
          \\
          \\
          =\dfrac{d}{dx}\left[x^{-v}\sum\limits_{n=0}^{\infty} \dfrac{(-1)^n}{n!\Gamma\left(1+v+n\right) }(\dfrac{x}{2})^{2n+v}\right]=\sum\limits_{n=0}^{\infty} \dfrac{(-1)^n}{n!\Gamma\left(1+v+n\right)} \dfrac{d}{dx}\left(\dfrac{x^{2n}}{2^{2n+v}}\right) \\
          \\
          \\
          =\sum\limits_{n=0}^{\infty} \dfrac{(-1)^n}{n!\Gamma\left(1+v+n\right)} \dfrac{1}{2^{2n+v}}\dfrac{d}{dx}\left(x^{2n}\right)=\sum\limits_{n=0}^{\infty} \dfrac{(-1)^n}{n!\Gamma\left(1+v+n\right)} \dfrac{2n}{2^{2n+v}}x^{2n-1} \\
          \\
          \\
          =\sum\limits_{n=0}^{\infty} \dfrac{(-1)^n}{n!\Gamma\left(1+v+n\right)} \dfrac{n}{2^{2n+v-1}}x^{2n-1}=\sum\limits_{n=0}^{\infty} \dfrac{(-1)^n}{(n-1)!\Gamma\left(1+v+n\right)} \dfrac{1}{2^{2n+v-1}}x^{2n-1} \\
          \\
          \\
          =x^{-v}\sum\limits_{n=0}^{\infty} \dfrac{(-1)^n}{(n-1)!\Gamma\left(1+v+n\right)} \left(\dfrac{x}{2}\right)^{2n+v-1} \\
          \\
          \\
          =x^{-v}\sum\limits_{n=0}^{\infty} \dfrac{(-1)^n}{((u+1)-1)!\Gamma\left(1+v+(u+1)\right)} \left(\dfrac{x}{2}\right)^{2(u+1)+v-1}=-x^{-v}  \mathcal{J}_{v+1}(x) \\
          \\
          \\
          \\
          \\
          \therefore ~~~  \dfrac{d}{dx} \left(x^{-v} \mathcal{J}_{v}(x)\right)=-x^{-v} \mathcal{J}_{v+1}(x)
        $
      }


    \item Using a graphing tool of your choice (e.g., Wolfram alpha, where all you need is a browser), plot $J_4$ and $J_5$, and verify that the $n-$th root of $J_5$ is always larger than the $n-$th root of $J_4$ (for the same $n$; do not count $x=0$ as a root).  In other words, that $\zeta_{4n}<\zeta_{5n}$ for all $n$. (include the graph in your submission; it will suffice to check the result for $n=1,2,3,4$). 

    \includegraphics[height=6cm, width=9cm]{bessel.JPG}

      \textcolor{hwColor}{
        The first few zeroes of Bessel functions are tabulated below. \\ \\
        \includegraphics[height=3cm, width=9cm]{table1.JPG}
      }

  \end{enumerate}

\end{document}
