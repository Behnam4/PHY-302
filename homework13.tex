\documentclass[fleqn]{article}
\oddsidemargin 0.0in
\textwidth 6.0in
\thispagestyle{empty}
\usepackage{import}
\usepackage{amsmath}
\usepackage{graphicx}
\usepackage{flexisym}
\usepackage{amssymb}
\usepackage{bigints} 
\usepackage[english]{babel}
\usepackage[utf8x]{inputenc}
\usepackage{float}
\usepackage[colorinlistoftodos]{todonotes}

\definecolor{hwColor}{HTML}{AD53BA}

\begin{document}

  \begin{titlepage}

    \newcommand{\HRule}{\rule{\linewidth}{0.5mm}}

    \center


    \textsc{\LARGE Arizona State University}\\[1.5cm]

    \textsc{\LARGE Mathematical Methods For Physics II }\\[1.5cm]


    \begin{figure}
      \includegraphics[width=\linewidth]{asu.png}
    \end{figure}


    \HRule \\[0.4cm]
    { \huge \bfseries Homework Thirteen}\\[0.4cm] 
    \HRule \\[1.5cm]

    \textbf{Behnam Amiri}

    \bigbreak

    \textbf{Prof: Cecilia Lunardini}

    \bigbreak


    \textbf{{\large \today}\\[2cm]}

    \vfill

  \end{titlepage}

  \textbf{Part A}
  \begin{enumerate}
    \item Let ${\mathcal C}$ be a unit circle centered at $z = 1$. Compute
      $$
      \oint_{\mathcal C}   \frac{z^2 + 1}{z^2 - 1} dz
      $$
      (Hint: write the integral as a sum of terms using partial fractions; be careful as of which poles are within the given contour). 

        \textcolor{hwColor}{
          We can start by doing a long-division since the the degree of the numerator is not 
          less than degree of the denominator.\\
          \\
          \\
          $
            \dfrac{z^2+1}{z^2-1}=\dfrac{2}{z^2-1}+1 \\
            \\
            \\
            \dfrac{2}{z^2-1}=\dfrac{2}{(z-1)(z+1)}=\dfrac{A}{z+1}+\dfrac{B}{z-1}=\dfrac{A(x-1)+B(x+1)}{(z+1)(z-1)},
            ~~~~~ \therefore ~~~ \begin{cases}
              A=-1 \\
              B=1
            \end{cases} \\
            \\
            \\
            \Longrightarrow ~~~ \dfrac{z^2+1}{z^2-1}=1-\dfrac{1}{z+1}+\dfrac{1}{z-1} ~~~ \surd \\
          $
          \\
          \\
          Now let's rewrite the given integral
          \\
          \\
          $$
            \oint\limits_{\mathcal C} \frac{z^2 + 1}{z^2 - 1} dz \\ \\
            =\oint\limits_{\mathcal C} \left(1-\dfrac{1}{z+1}+\dfrac{1}{z-1}\right) dz \\ \\
            =\oint\limits_{\mathcal C} dz-\oint\limits_{\mathcal C} \dfrac{1}{z+1}dz+\oint\limits_{\mathcal C}\dfrac{1}{z-1}dz=I_A+I_B+I_C
          $$
          Let's review what the Cauchy's theorem is. Suppose $f(z)$ is a complex function and $\mathcal{C}$ is a closed 
          curve in the complex plane. If the following conditions are satisfied:
          \begin{itemize}
            \item $f(z)$ is holomorphic on and everywhere inside $\mathcal{C}$.
            \item $\mathcal{C}$ is a simple curve (does not cross itself).
            \item $\mathcal{C}$ has a finite number of corners.
          \end{itemize}
          then we have 
          $$\oint\limits_{\mathcal{C}} f(z) dz=0$$
          We know that for this problem $\mathcal{C}$ is a unit circle centered at $z=1$. And we have $-1$ and $1$ as 
          our singularities for the given function. If we look carefully we can see that only $I_B$ in inside $\mathcal{C}$,
          hence we can equal to to zero. \\
          \\
          $
            I_C=\oint\limits_{\mathcal{C}} \dfrac{1}{z-1}dz \Longrightarrow \begin{cases}
              f(z)=1 \\
              \\
              z_0=1 
            \end{cases}
            \Longrightarrow f(1)=\dfrac{1}{2 \pi i} \oint\limits_{\mathcal{C}} \dfrac{1}{z-1}=1 \\
            \\
            \\
            \therefore ~~~ \oint\limits_{\mathcal{C}} \dfrac{1}{z-1}=2 \pi i ~~~ \surd
          $ 
          \\
          \\
          For $I_A$ we have $\oint\limits_{\mathcal C} dz=\oint\limits_{0}^{2 \pi} i e^{it} dt=1-1=0$. \\
          \\
          \\
          \\
          $
            \therefore ~~~ \oint\limits_{\mathcal C} \dfrac{z^2+1}{z^2-1} dz=I_A+I_B+I_C=2 \pi i ~~~ \surd
          $
          \\
          \\
          \\
        }

      \item Compute the same integral as above, where $\mathcal{C}$ is a circle of radius 2 centered at $z=0$.
      
        \textcolor{hwColor}{
          Since this question is the same as the previous one we have
          $$
            \oint\limits_{\mathcal{C}} \dfrac{z^2+1}{z^2-1} dz=\oint\limits_{\mathcal C} dz-\oint\limits_{\mathcal C} \dfrac{1}{z+1}dz+\oint\limits_{\mathcal C}\dfrac{1}{z-1}dz=I_A+I_B+I_C
          $$
          For this problem $\mathcal{C}$ is a circle of radius 2 centered at $z=0$. Therefore, both $I_B$ and $I_C$ 
          are inside $\mathcal{C}$, hence: \\
          \\
          $
            \oint\limits_{\mathcal C} \dfrac{z^2+1}{z^2-1} dz=2 \pi i-2\pi i=0 ~~~ \surd
          $ 
          \\
          \\
        }

      \item Use the generalized Cauchy's formula to evaluate
      $$
      \oint_{\mathcal C}   \frac{\sinh z}{z^4} dz
      $$
      where ${\mathcal C}$ is the unit circle centered at the origin.


        \textcolor{hwColor}{
          \\
          \\
          Let us remember that the generalized Cauchy formula allows us to write derivatives as integrals:
          $$\oint\limits_{\mathcal C} \dfrac{f(z)}{(z-z_0)^{n+1}}dz=\dfrac{2 \pi i}{n!} \dfrac{\partial^n f}{\partial z^n} \Big|_{z=z_0}$$
          We are told that $\mathcal{C}$ is the unit circle centered at the origin, therefore $\mathcal{C}:|z|=1$. 
          \\
          \\
          \textbf{Method I:} \\
          \\
          Since we divide by $z^4$, we need the third derivative in Cauchy's formula \\ \\
          $
            \oint\limits_{\mathcal C} \dfrac{sinh(z)}{z^4} dz=\dfrac{2 \pi i}{3!} \dfrac{\partial^3}{\partial z^3} \left(sinh(z)\right) \Big|_{z=0}
            =\dfrac{2 \pi i}{3!} \dfrac{\partial^3}{\partial z^3} \left(sinh(z)\right) \Big|_{z=0}
            =\dfrac{2 \pi i}{3!} cosh(z) \Big|_{z=0}
            \\
            \\
            \\
            \therefore ~~~ \oint\limits_{\mathcal C} \dfrac{sinh(z)}{z^4} dz=i\dfrac{\pi}{3} ~~~~ \surd
          $ 
          \\
          \\
          \rule{15cm}{1pt}
          \\
          \\
          \textbf{Method II:} \\
          \\
          Since $z=0$ is a pole of order 3 of $f(z)=\dfrac{sinh(z)}{z^4}$, we have \\
          \\
          \\
          $
            Res_{z=0} \dfrac{sinh(z)}{z^4}=\dfrac{1}{2!} \lim\limits_{z\to\infty} \dfrac{d^2}{dz^2} \left(z^3 \dfrac{sinh(z)}{z^4}\right)
            =\dfrac{1}{2!} \lim\limits_{z\to\infty} \dfrac{d^2}{dz^2} \left(\dfrac{sinh(z)}{z}\right) \\
            \\
            \\
            =\dfrac{1}{2!} \lim\limits_{z\to\infty} \left[\dfrac{z^2 sinh(z)}{z^3}-\dfrac{2 \left(z cosh(z)-sinh(z)\right)}{z^3}\right]
          $
          \\
          \\
          Using L'H$\hat{o}$pital's rule: \\
          \\
          $
            Res_{z=0} \dfrac{sinh(z)}{z^4}=\dfrac{1}{2!} \lim\limits_{z\to\infty} \left[\dfrac{2z sinh(z)+z^2 cosh(z)-2\left(cosh(z)+z sinh(z)-cosh(z)\right)}{3z^2}\right] \\
            \\
            \\
            =\dfrac{1}{2!} \lim\limits_{z\to\infty} \dfrac{z^2 cosh(z)}{3z^2}=\dfrac{1}{2!} \lim\limits_{z\to\infty} \dfrac{2z cosh(z)+z^2 sinh(z)}{6z} \\
            \\
            \\
            =\dfrac{1}{2!} \lim\limits_{z\to\infty} \dfrac{2 cosh(z)+2z sinh(z)+2z sinh(z)+z^2 cosh(z)}{6} \\
            \\
            \\
            \\
            Res_{z=0} \dfrac{sinh(z)}{z^4}=\dfrac{1}{6} ~~~ \surd
            \\
            \\
            \\
            \oint\limits_{\mathcal C} \dfrac{sinh(z)}{z^4} dz=2\pi i Res_{z=0} \dfrac{sinh(z)}{z^4}=2 \pi i \left(\dfrac{1}{6}\right) \\
            \\
            \\
            \\
            \therefore ~~~ \oint\limits_{\mathcal C} \dfrac{sinh(z)}{z^4} dz=i\dfrac{\pi}{3} ~~~~ \surd
          $
        }


  \end{enumerate}

  \pagebreak

  \textbf{Part B}
  \begin{enumerate}
    % Exercises 30
    \item Prove the identity:
    $$\frac{1}{1-\omega} = 1 + \omega +\omega^2 + \omega^3 + ...... + \omega^{n-1}  + \frac{\omega^n}{1- \omega}~,$$
    where the last term is called the remainder of the series. (Hint: it may be easier to prove the equivalent expression: $(1- \omega^n) =(1- \omega)( 1 + \omega +\omega^2 + \omega^3 + ...... + \omega^{n-1} )$~~).

    % Exercises 31 + older lecture material.
    \item  Use the result of the previous exercise to prove the expression of the remainder of a generic Taylor series, $R_n$ (see theory segment 35).  In other words, prove that, if the conditions for Taylor's theorem are satisfied, the following equality holds: 
    $$\sum^{\infty}_{k=n} \frac{1}{k!} d^{(k)} f(z_0) (z-z_0)^k = \frac{(z-z_0)^n}{2 \pi i} \oint_{\mathcal C}  \frac{f(z^\prime)}{(z^\prime - z_0)^n(z^\prime - z)} dz^\prime ~,$$
    where $d^{(k)} f(z_0)$ is the $k$-th order derivative of $f(z)$, evaluated in the point $z=z_0$.   (Hint: take inspiration from the derivation of the expression of the Taylor's series expansion done in Theory segment 35). 
    
  \end{enumerate}

\end{document}
