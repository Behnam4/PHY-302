\documentclass[fleqn]{article}
\oddsidemargin 0.0in
\textwidth 6.0in
\thispagestyle{empty}
\usepackage{import}
\usepackage{amsmath}
\usepackage{graphicx}
\usepackage{flexisym}
\usepackage{amssymb}
\usepackage{bigints} 
\usepackage[english]{babel}
\usepackage[utf8x]{inputenc}
\usepackage{float}
\usepackage[colorinlistoftodos]{todonotes}

\definecolor{hwColor}{HTML}{AD53BA}

\begin{document}

  \begin{titlepage}

    \newcommand{\HRule}{\rule{\linewidth}{0.5mm}}

    \center



    \textsc{\LARGE Arizona State University}\\[1.5cm]

    \textsc{\LARGE Mathematical Methods For Physics II }\\[1.5cm]


    \begin{figure}
      \includegraphics[width=\linewidth]{asu.png}
    \end{figure}


    \HRule \\[0.4cm]
    { \huge \bfseries Homework Fifteen}\\[0.4cm] 
    \HRule \\[1.5cm]

    \textbf{Behnam Amiri}

    \bigbreak

    \textbf{Prof: Cecilia Lunardini}

    \bigbreak


    \textbf{{\large \today}\\[2cm]}

    \vfill

  \end{titlepage}

  \begin{enumerate}
    \item Evaluate the following definite integral: 
    $$\int^{2\pi}_0 \frac{d\theta}{p^2 -2p \cos\theta+1}~$$ 
    where $-1<p<1$.

      \textcolor{hwColor}{
        $
          \begin{cases}
            z=e^{i \theta} \longrightarrow dz=iz d\theta
            \\
            \\
            e^{i \theta}+e^{-i\theta}=\left[cos(\theta)+i sin(\theta)\right]+\left[cos(\theta)-i sin(\theta)\right] \Longrightarrow cos(\theta)=\dfrac{1}{2} \left(e^{i \theta}+e^{-i \theta}\right)=\dfrac{1}{2} \left(z+\dfrac{1}{z}\right)
          \end{cases}
          \\
          \\
          \\
          I=\bigints\limits_{0}^{2 \pi} \dfrac{d\theta}{p^2-2p cos(\theta)+1}
          \\
          \\
          =\bigints\limits_{0}^{2 \pi} \dfrac{\dfrac{dz}{iz}}{p^2-2p \left[\dfrac{1}{2} \left(z+\dfrac{1}{z}\right)\right]+1}
          \\
          \\
          =\dfrac{1}{i} \bigints\limits_{0}^{2 \pi} \dfrac{dz}{zp^2-p\left(z^2+1\right)+z}
          \\
          \\
          =-i \bigints\limits_{0}^{2 \pi} \dfrac{dz}{zp^2-p\left(z^2+1\right)+z}
          \\
          \\
          =-i \bigints\limits_{0}^{2 \pi} \dfrac{dz}{(p-z)(zp-1)}
        $
        \\
        \\
        What we ended up with is a function that has two poles 
        \\
        \\
        $
          \begin{cases}
            z=p
            \\
            z=p^{-1}
          \end{cases}
        $
        Based on the two poles we have the two scenarios: \\
        \\
        \\
        \\
        \textbf{(A) ~ $|z|>1$}
        \\
        \\
        $
          Res(f(p^{-1}))=\lim\limits_{z \to p^{-1}} \dfrac{i (z-p^{-1})}{p(z-p)(z-p^{-1})}=i\dfrac{1}{1-p^2}
        $
        \\
        \\
        \rule{15cm}{1pt}
        \\
        \\
        \textbf{(B) ~ $|z|<1$}
        \\
        \\
        $
          Res(f(p))=\lim\limits_{z \to p} \dfrac{i (z-p)}{p(z-p)(z-p^{-1})}=\dfrac{i}{p^2-1}
        $
        \\
        \\
        \\
        I=$
          \begin{cases}
            I_A=2 \pi i Res(f(p^{-1}))=\dfrac{2 \pi}{1-p^2} ~~~~~ \surd
            \\
            \\
            I_B=2 \pi i Res(f(p))=\dfrac{2 \pi}{p^2-1} ~~~~~ \surd
          \end{cases}
        $
      }
    
    \item Evaluate the following integrals:
    \begin{enumerate}
      \item $$ \int^{+\infty}_{-\infty} \frac{ e^{imx}}{1+x^2}dx $$ 
      \item $$ \int^{+\infty}_{0} \frac{ x^2}{1+x^4}dx$$
      
      \textcolor{hwColor}{
        The answer should be $\pi e^k$
      }
    \end{enumerate}
    (Hint: for the (a), use the property $\displaystyle\lim_{|z|  \to \infty} |e^{im z}| = \displaystyle\lim_{y  \to \infty} e^{-m y} = 0 $.  For (b): note that the integrand is an even function. ) 
    
    \item Evaluate the integral:
    $$\int^{+\infty}_{0} \frac{ \sqrt{x}}{1+x^3}dx$$
    (Hint: take inspiration from the example(s) shown in class. ) 
    
  \end{enumerate}

\end{document}
