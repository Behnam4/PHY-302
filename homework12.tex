\documentclass[fleqn]{article}
\oddsidemargin 0.0in
\textwidth 6.0in
\thispagestyle{empty}
\usepackage{import}
\usepackage{amsmath}
\usepackage{graphicx}
\usepackage{flexisym}
\usepackage{amssymb}
\usepackage{bigints} 
\usepackage[english]{babel}
\usepackage[utf8x]{inputenc}
\usepackage{float}
\usepackage[colorinlistoftodos]{todonotes}

\definecolor{hwColor}{HTML}{AD53BA}

\begin{document}

  \begin{titlepage}

    \newcommand{\HRule}{\rule{\linewidth}{0.5mm}}

    \center



    \textsc{\LARGE Arizona State University}\\[1.5cm]

    \textsc{\LARGE Mathematical Methods For Physics II }\\[1.5cm]


    \begin{figure}
      \includegraphics[width=\linewidth]{asu.png}
    \end{figure}


    \HRule \\[0.4cm]
    { \huge \bfseries Homework Twelve}\\[0.4cm] 
    \HRule \\[1.5cm]

    \textbf{Behnam Amiri}

    \bigbreak

    \textbf{Prof: Cecilia Lunardini}

    \bigbreak


    \textbf{{\large \today}\\[2cm]}

    \vfill

  \end{titlepage}

  \textbf{Part A}
  \begin{enumerate}

    \item Find the poles in the following functions and determine their orders:
      \begin{enumerate}
        \item  $\frac{1}{z^2 - 2z + 1}$ , 
        \item  $\frac{1}{z^2 - 2z - 1}$ , 
        \item  $\frac{e^{(z-1)} - 1}{(z^2 + 1) (z - 1)}$  (Hint: be careful when analyzing the behavior at $z=1$, does the function diverge in that point? )
      \end{enumerate}

    \item Show that, if $u$ and $v$ are the real and imaginary parts of the analytic function, f$ (z)$,
    the family of curves, $u(x, y) = $constant, is orthogonal to the family of curves, $v(x, y) =$ constant.\\
    (hint: Use the Cauchy-Riemann conditions to show that the slope of the curve $u = c$ is the negative inverse of the slope of the curve $v = d$ (with $d$ and $c$ being constants) at the same point $(x, y)$). 

    \textcolor{hwColor}{
      \\
      From lecture and from page 826 of the textbook, we saw that we defined a function as analytic 
      if it was single-valued and differentiable at all points in a domain $\mathcal{R}$ where we defined differentiability 
      as the unique existence of the following expression (equation 24.1 in the textbook): \\
      \\
      $$f(z)=\lim\limits_{\Delta z \to 0} \left[\dfrac{f(z+\Delta z)-f(z)}{\Delta z}\right]$$ \\
      \\
      Cauchy–Riemann equations imply orthogonal gradients. For a real form $f=u+i v$, If $u$ and $v$ satisfy 
      the Cauchy–Riemann equations, then the two gradient vectors have dot product as follows: \\
      \\
      $$\nabla u. \nabla v
      =\dfrac{\partial u}{\partial x} \dfrac{\partial v}{\partial x}+\dfrac{\partial u}{\partial y} \dfrac{\partial v}{\partial y}
      =\dfrac{\partial u}{\partial y} \dfrac{\partial u}{\partial x}-\dfrac{\partial u}{\partial x}\dfrac{\partial u}{\partial y}
      =0$$
      \\
      Where in the last line we have used the Cauchy-Riemann relations to rewrite the partial 
      deriviatives of $v$ as partial deriviatives of $u$. Since the scalar product of the 
      normal vector is zero. they must be orthogonal, and the curves $u(x,y)=C$ and $v(x,y)=C$ must
      therefore intersect at right angles. \\
      \\
      Deduction: The level curves of $u$ and the level curves of $v$ are orthogonal to each other.
      (In real calculus, such families of curves are known as orthogonal trajectories.)
    }

  \end{enumerate}

  \pagebreak

  \textbf{Part B}

    \textcolor{hwColor}{
      \\
      \\
      The Taylor series of a function is an infinite sum of terms that are expressed in terms of the function's 
      derivatives at a single point. For most common functions, the function and the sum of its Taylor series are equal near this point. \\
      \\
      The Taylor series of a real or complex-valued function $f(x)$ that is infinitely differentiable 
      at a real or complex number a is the power series 
      $$f(x)=\sum\limits_{n=0}^{\infty} \dfrac{f^{(n)}(a)}{n!} \left(x-a\right)^n=f(a)+ \dfrac{f^'(a)}{1!} \left(x-a\right)+\dfrac{f^{''}(a)}{2!} \left(x-a\right)^2+\dfrac{f^{'''}(a)}{3!} \left(x-a\right)^3+...$$
    }

  \begin{enumerate}
    \item Expand the function 
    $$f(z)= \frac{1}{z+1}$$ as a Taylor series about the point $z_0=0$ 
    (i.e. find the Maclaurin expansion.). Find the region of convergence.

    \textcolor{hwColor}{
      $
        f(z)=f(z_0)+   f^'(z_0)(z-z_0)    +\dfrac{f^{''}(z_0)}{2!} (z-z_0)^2  +   \dfrac{f^{'''}(z_0)}{3!} (z-z_0)^3+... \\
        \\
        \\
        =f(0)+ f^'(0) z   +   \dfrac{f^{''}(0)}{2!} z^2+  \dfrac{f^{'''}(0)}{3!} z^3+... \\
        \\
        \\
        =1+\dfrac{-1}{(0+1)^2}z+\dfrac{\dfrac{2}{(0+1)^3}}{2!} z^2+ \dfrac{-\dfrac{6}{(0+1)^4}}{3!}  z^3 \\
        \\
        \\
        =1-z+z^2-z^3+...
        \\
        \\
        \\
        \therefore ~~~ f(z)=\sum\limits_{n=0}^{\infty} (-1)^n ~ z^n
      $
    }

    \item Expand the same function of the previous exercise as a Taylor series about $z_0 = 1$ 
    and find the region of convergence.

      \textcolor{hwColor}{
        Behnam was here
      }

    \item Expand the function
    $$f(z) = \frac{z + 1}{z^2 - z - 6}$$
    as a Taylor series to order $(z - 2)^3$ about the point $z = 2$. Find the region of convergence. (Hint: you can rewrite the fiven function as a sum of two terms, using partial fractions). 

  \end{enumerate}

\end{document}
