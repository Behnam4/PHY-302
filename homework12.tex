\documentclass[fleqn]{article}
\oddsidemargin 0.0in
\textwidth 6.0in
\thispagestyle{empty}
\usepackage{import}
\usepackage{amsmath}
\usepackage{graphicx}
\usepackage{flexisym}
\usepackage{amssymb}
\usepackage{bigints} 
\usepackage[english]{babel}
\usepackage[utf8x]{inputenc}
\usepackage{float}
\usepackage[colorinlistoftodos]{todonotes}

\definecolor{hwColor}{HTML}{AD53BA}

\begin{document}

  \begin{titlepage}

    \newcommand{\HRule}{\rule{\linewidth}{0.5mm}}

    \center



    \textsc{\LARGE Arizona State University}\\[1.5cm]

    \textsc{\LARGE Mathematical Methods For Physics II }\\[1.5cm]


    \begin{figure}
      \includegraphics[width=\linewidth]{asu.png}
    \end{figure}


    \HRule \\[0.4cm]
    { \huge \bfseries Homework Twelve}\\[0.4cm] 
    \HRule \\[1.5cm]

    \textbf{Behnam Amiri}

    \bigbreak

    \textbf{Prof: Cecilia Lunardini}

    \bigbreak


    \textbf{{\large \today}\\[2cm]}

    \vfill

  \end{titlepage}

  \textbf{Part A}
  \begin{enumerate}

    \item Find the poles in the following functions and determine their orders:
    
      \textcolor{hwColor}{
        As you saw from lecture and from your readings, the singularitieswill occur at points that cause the 
        denominator to become zero.
        $$a=\lim\limits_{z \to z_0} f(z) (z-z_0)^n$$  
      }

      \begin{enumerate}
        \item  $\frac{1}{z^2 - 2z + 1}$ 

          \textcolor{hwColor}{
            $
              f(z)=\dfrac{1}{(z-1)^2} \\
              \\
              \\
              \lim\limits_{z \to z_0} f(z) (z-z_0)^n=\lim\limits_{z \to 1} \dfrac{1}{(z-1)^2} (z-1)^2 \\
              \\
              \\
              \therefore ~~~ \lim\limits_{z \to 1}=1 ~~~ \surd
            $ \\ \\
            Based on $f(z)$ we assumed the pole of the function is $z=1$. Hence, $1$ is pole of order 2. \\ \\
          }

        \item  $\frac{1}{z^2 - 2z - 1}$ 

          \textcolor{hwColor}{
            Let's find the roots of the denominator of $f(z)$
            $
              z=\dfrac{2\pm \sqrt{4-4(1)(-1)}}{2} \Rightarrow z=1+\pm \sqrt{2} \\
              \\
              \\
              \lim\limits_{z \to z_0} f(z) (z-z_0)^n=\lim\limits_{z \to (1+\sqrt{2})} \dfrac{1}{ \left[z-(1+\sqrt{2})\right] \left[z-(1-\sqrt{2})\right]} \left[z-(1+\sqrt{2})\right]=\dfrac{1}{2\sqrt{2}} \\
              \\
              \\
              \lim\limits_{z \to z_0} f(z) (z-z_0)^n=\lim\limits_{z \to (1-\sqrt{2})} \dfrac{1}{ \left[z-(1+\sqrt{2})\right] \left[z-(1-\sqrt{2})\right]} \left[z-(1-\sqrt{2})\right]=-\dfrac{1}{2\sqrt{2}} \\
              \\
              \\
              \\
              \therefore ~~~~ \begin{cases}
                1+\sqrt{2} \\
                \\
                1-\sqrt{2}
              \end{cases}
            $ order 1 poles at these points. \\ \\
          }

        \item  $\frac{e^{(z-1)} - 1}{(z^2 + 1) (z - 1)}$  (Hint: be careful when analyzing the behavior at $z=1$, does the function diverge in that point? )

          \textcolor{hwColor}{
            $
              \lim\limits_{z \to z_0} f(z) (z-z_0)^n=\lim\limits_{z \to i} \dfrac{e^{z-1}-1}{(z^2+1)(z-1)} (z-i)=\lim\limits_{z \to i} \dfrac{e^{z-1}-1}{(z-1)(z+i)(z-i)}(z-i) \\
              \\
              \\
              =\dfrac{e^{i-1}-1}{(i-1)(i+i)} \\
              \\
              =\dfrac{e^{i-1}-1}{(i-1)2i} \\
              \\
              =\dfrac{e^{i-1}-1}{(2i^2-2i)} \\
              \\
              =-\dfrac{e^{i-1}-1}{2(1+i)} \\
              \\
              =-\dfrac{e^{i-1}-1}{2(1+i)} \times \dfrac{1-i}{1-i} \\
              \\
              \\
              \therefore ~~~ \lim\limits_{z \to i} \dfrac{e^{z-1}-1}{(z-1)(z+i)(z-i)}(z-i)=-\dfrac{(e^{i-1}-1)(1-i)}{4} 
              \\
              \\
              \rule{15cm}{1pt}
              \\
              \\
              \lim\limits_{z \to z_0} f(z) (z-z_0)^n=\lim\limits_{z \to -i} \dfrac{e^{z-1}-1}{(z^2+1)(z-1)} (z+i)=\lim\limits_{z \to -i} \dfrac{e^{z-1}-1}{(z-1)(z+i)(z-i)} (z+i) \\
              \\
              \\
              =\dfrac{e^{-i-1}-1}{(-i-1)(-i-i)} \\
              \\
              =\dfrac{e^{-i-1}-1}{-2i(-i-1)} \\
              \\
              =\dfrac{e^{-i-1}-1}{2i(i+1)} \\
              \\
              =\dfrac{e^{-i-1}-1}{2(i-1)} \\
              \\
              =\dfrac{e^{-i-1}-1}{2(i-1)} \times \dfrac{1+i}{1+i} \\
              \\
              \\
              \therefore ~~~ \lim\limits_{z \to -i} \dfrac{e^{z-1}-1}{(z^2+1)(z-1)} (z+i)=-\dfrac{(e^{-i-1}-1)(1+i)}{4} 
              \\
              \\
              \rule{15cm}{1pt}
              \\
              \\
              \lim\limits_{z \to z_0} f(z) (z-z_0)^n=\lim\limits_{z \to 1} \dfrac{e^{z-1}-1}{(z^2+1)(z-1)} (z-1)=0
            $ \\
            \\
            \\
            Hence, we have poles of order one at the points $i$ and $-i$. \\ \\
          }

      \end{enumerate}

    \item Show that, if $u$ and $v$ are the real and imaginary parts of the analytic function, f$ (z)$,
    the family of curves, $u(x, y) = $constant, is orthogonal to the family of curves, $v(x, y) =$ constant.\\
    (hint: Use the Cauchy-Riemann conditions to show that the slope of the curve $u = c$ is the negative inverse of the slope of the curve $v = d$ (with $d$ and $c$ being constants) at the same point $(x, y)$). 

    \textcolor{hwColor}{
      \\
      From lecture and from page 826 of the textbook, we saw that we defined a function as analytic 
      if it was single-valued and differentiable at all points in a domain $\mathcal{R}$ where we defined differentiability 
      as the unique existence of the following expression (equation 24.1 in the textbook): \\
      \\
      $$f(z)=\lim\limits_{\Delta z \to 0} \left[\dfrac{f(z+\Delta z)-f(z)}{\Delta z}\right]$$ \\
      \\
      Cauchy–Riemann equations imply orthogonal gradients. For a real form $f=u+i v$, If $u$ and $v$ satisfy 
      the Cauchy–Riemann equations, then the two gradient vectors have dot product as follows: \\
      \\
      $$\nabla u. \nabla v
      =\dfrac{\partial u}{\partial x} \dfrac{\partial v}{\partial x}+\dfrac{\partial u}{\partial y} \dfrac{\partial v}{\partial y}
      =\dfrac{\partial u}{\partial y} \dfrac{\partial u}{\partial x}-\dfrac{\partial u}{\partial x}\dfrac{\partial u}{\partial y}
      =0$$
      \\
      Where in the last line we have used the Cauchy-Riemann relations to rewrite the partial 
      deriviatives of $v$ as partial deriviatives of $u$. Since the scalar product of the 
      normal vector is zero. they must be orthogonal, and the curves $u(x,y)=C$ and $v(x,y)=C$ must
      therefore intersect at right angles. \\
      \\
      Deduction: The level curves of $u$ and the level curves of $v$ are orthogonal to each other.
      (In real calculus, such families of curves are known as orthogonal trajectories.) 
      \\
      \\
      \\
      Slope of a tangent to curves $u(x,y)=u_0$ and $v(x,y)=v_0$
      $
        \begin{cases}
          \dfrac{dy}{dx}=-\dfrac{\partial u/\partial x}{\partial u/\partial y} \\
          \\
          \dfrac{dy}{dx}=-\dfrac{\partial v/\partial x}{\partial v/\partial y}
        \end{cases}
      $ \\ 
      \\
      At $(x_0, y_0)$ the Cauchy-Riemann equations $u_x=v_y$ and $u_y=-v_x$. Then the product 
      of the two slope functions is \\ \\ 
      $
        \left(-\dfrac{\partial u/\partial x}{\partial u/\partial y}\right)  \left(-\dfrac{\partial v/\partial x}{\partial v/\partial y}\right)=-1
      $ 
    }

  \end{enumerate}

  \pagebreak

  \textbf{Part B}

    \textcolor{hwColor}{
      \\
      \\
      The Taylor series of a function is an infinite sum of terms that are expressed in terms of the function's 
      derivatives at a single point. For most common functions, the function and the sum of its Taylor series are equal near this point. \\
      \\
      The Taylor series of a real or complex-valued function $f(x)$ that is infinitely differentiable 
      at a real or complex number a is the power series 
      $$f(x)=\sum\limits_{n=0}^{\infty} \dfrac{f^{(n)}(a)}{n!} \left(x-a\right)^n=f(a)+ \dfrac{f^'(a)}{1!} \left(x-a\right)+\dfrac{f^{''}(a)}{2!} \left(x-a\right)^2+\dfrac{f^{'''}(a)}{3!} \left(x-a\right)^3+...$$
    }

  \begin{enumerate}
    \item Expand the function 
    $$f(z)= \frac{1}{z+1}$$ as a Taylor series about the point $z_0=0$ 
    (i.e. find the Maclaurin expansion.). Find the region of convergence.

    \textcolor{hwColor}{
      $
        f(z)=f(z_0)+f^'(z_0)(z-z_0)+\dfrac{f^{''}(z_0)}{2!} (z-z_0)^2+\dfrac{f^{'''}(z_0)}{3!} (z-z_0)^3+... \\
        \\
        \\
        =f(0)+ f^'(0) z+ \dfrac{f^{''}(0)}{2!} z^2+\dfrac{f^{'''}(0)}{3!} z^3+... \\
        \\
        \\
        =1+\dfrac{-1}{(0+1)^2}z+\dfrac{\dfrac{2}{(0+1)^3}}{2!} z^2+ \dfrac{-\dfrac{6}{(0+1)^4}}{3!}  z^3 \\
        \\
        \\
        =1-z+z^2-z^3+...
        \\
        \\
        \\
        \therefore ~~~ f(z)=\sum\limits_{n=0}^{\infty} (-1)^n ~ z^n
      $
      \\
      \\
      From equation 24.14 of the textbook, we see that we can also use the ratio test to determine
      the radius of convergence of a series: \\ \\
      $
        \lim\limits_{n \to \infty} \dfrac{|a_{n+1}|}{a_n}=\lim\limits_{n \to \infty} \dfrac{|(-1)^{n+1}|}{(-1)^n} \\
        \\
        \\
        \therefore ~~~ \lim\limits_{n \to \infty} \dfrac{|a_{n+1}|}{a_n}=1 ~~ \surd
      $
      \\
      \\
      Making a circle where the series is centered and expand until we hit a singularity, hence the 
      region of convergence is inside a circle centered at $z_0=0$ with a radius of 1. $(z+1)^{-1}$ has a singularity at $-1$.
    }

    \item Expand the same function of the previous exercise as a Taylor series about $z_0 = 1$ 
    and find the region of convergence.

      \textcolor{hwColor}{
        $
          f(z)=f(z_0)+f^'(z_0)(z-z_0)+\dfrac{f^{''}(z_0)}{2!} (z-z_0)^2+\dfrac{f^{'''}(z_0)}{3!} (z-z_0)^3+... \\
          \\
          \\
          =f(1)+f^'(1)(z-1)+\dfrac{f^{''}(1)}{2!} (z-1)^2+\dfrac{f^{'''}(1)}{3!} (z-1)^3+... \\ 
          \\
          \\
          =\dfrac{1}{2}-\dfrac{1}{4}(z-1)+\dfrac{\dfrac{2}{2^3}}{2!} (z-1)^2-\dfrac{\dfrac{6}{2^4}}{3!} (z-1)^3 \\
          \\
          \\
          \\
          \therefore ~~~ f(z)=\sum\limits_{n=0}^{\infty} \left(-\dfrac{1}{2}\right)^n (z-1)^n \\
          \\
          \\
          \\
          \dfrac{1}{R}=\lim\limits_{n \to \infty} |\dfrac{a_{n+1}}{a_n}|=\lim\limits_{n \to \infty} \Big|\dfrac{(-\dfrac{1}{2})^n+1}{(-\dfrac{1}{2})^n}\Big| \\
          \\
          \\
          \therefore ~~~ R=2  \\
        $
        \\
        The region of convergence is inside a circle centered a $1$ with a radius of $2$.
      }

    \item Expand the function
    $$f(z) = \frac{z + 1}{z^2 - z - 6}$$
    as a Taylor series to order $(z - 2)^3$ about the point $z = 2$. Find the region of convergence. (Hint: you can rewrite the fiven function as a sum of two terms, using partial fractions). 

    \textcolor{hwColor}{
      $
        f(z)=\dfrac{z+1}{z^2-z-6}=\dfrac{z+1}{(z+2)(z-3)} \Longrightarrow
        \dfrac{z+1}{(z+2)(z-3)}=\dfrac{A}{z+2}+\dfrac{B}{z-3} \\
        \\
        \\
        \therefore ~~~~ f(z)=\dfrac{1}{5(z+2)}+\dfrac{4}{5(z-3)} ~~~ \surd \\
        \\
        \\
        \begin{cases}
          f^'(z)=-\dfrac{1}{5(z+2)^2}-\dfrac{4}{5(z-3)^2} \\
          \\
          \\
          f^{''}(z)=\dfrac{2}{5(z+2)^3}+\dfrac{8}{5(z-3)^3} \\
          \\
          \\
          f^{'''}(z)=-\dfrac{6}{5(z+2)^4}-\dfrac{24}{5(z-3)^4}
        \end{cases} \\
        \\
        \\
        \begin{cases}
          f(2)=\dfrac{2+1}{(2+2)(2-3)}=-\dfrac{3}{4} \\
          \\
          \\
          f^'(2)=-\dfrac{1}{5(2+2)^2}-\dfrac{4}{5(2-3)^2}=-\dfrac{13}{16} \\
          \\
          \\
          f^{''}(2)=\dfrac{2}{5(2+2)^3}+\dfrac{8}{5(2-3)^3}=-\dfrac{51}{32} \\
          \\
          \\
          f^{'''}(2)=-\dfrac{6}{5(2+2)^4}-\dfrac{24}{5(2-3)^4}=-\dfrac{615}{128}
        \end{cases} \\
        \\
        \\
      $
      By plugging in the above values into the folowing equations we get: \\
      \\
      \\
      $
        f(z)=f(z_0)+f^'(z_0)(z-z_0)+\dfrac{f^{''}(z_0)}{2!} (z-z_0)^2+\dfrac{f^{'''}(z_0)}{3!} (z-z_0)^3+... \\
        \\
        \\
        f(z)=f(2)+f^'(2)(z-2)+\dfrac{f^{''}(2)}{2!} (z-2)^2+\dfrac{f^{'''}(2)}{3!} (z-2)^3+... \\
        \\
        \\
        =-\dfrac{3}{4}-\dfrac{13}{16} (z-2)-\dfrac{\dfrac{51}{32}}{2!} (z-2)^2-\dfrac{\dfrac{615}{128}}{3!} (z-2)^3+...
      $ \\
      \\
      \\
      The region of convergence is the inside of a circle centered at $z_0=2$ with radius of $3−z_0=1$.
    }

  \end{enumerate}

\end{document}
