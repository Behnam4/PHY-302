\documentclass[fleqn]{article}
\oddsidemargin 0.0in
\textwidth 6.0in
\thispagestyle{empty}
\usepackage{import}
\usepackage{amsmath}
\usepackage{graphicx}
\usepackage{flexisym}
\usepackage{amssymb}
\usepackage{bigints} 
\usepackage[english]{babel}
\usepackage[utf8x]{inputenc}
\usepackage{float}
\usepackage[colorinlistoftodos]{todonotes}

\definecolor{hwColor}{HTML}{AD53BA}

\begin{document}

  \begin{titlepage}

    \newcommand{\HRule}{\rule{\linewidth}{0.5mm}}

    \center



    \textsc{\LARGE Arizona State University}\\[1.5cm]

    \textsc{\LARGE Mathematical Methods For Physics II }\\[1.5cm]


    \begin{figure}
      \includegraphics[width=\linewidth]{asu.png}
    \end{figure}


    \HRule \\[0.4cm]
    { \huge \bfseries Homework Twelve}\\[0.4cm] 
    \HRule \\[1.5cm]

    \textbf{Behnam Amiri}

    \bigbreak

    \textbf{Prof: Cecilia Lunardini}

    \bigbreak


    \textbf{{\large \today}\\[2cm]}

    \vfill

  \end{titlepage}

  \textbf{Part A}
  \begin{enumerate}

    \item Find the poles in the following functions and determine their orders:
      \begin{enumerate}
        \item  $\frac{1}{z^2 - 2z + 1}$ , 
        \item  $\frac{1}{z^2 - 2z - 1}$ , 
        \item   $\frac{e^{(z-1)} - 1}{(z^2 + 1) (z - 1)}$  (Hint: be careful when analyzing the behavior at $z=1$, does the function diverge in that point? )
      \end{enumerate}

    \item Show that, if $u$ and $v$ are the real and imaginary parts of the analytic function, f$ (z)$,
    the family of curves, $u(x, y) = $constant, is orthogonal to the family of curves, $v(x, y) =$ constant.\\
    (hint: Use the Cauchy-Riemann conditions to show that the slope of the curve $u = c$ is the negative inverse of the slope of the curve $v = d$ (with $d$ and $c$ being constants) at the same point $(x, y)$). 

  \end{enumerate}

  \pagebreak

  \textbf{Part B}
  \begin{enumerate}
    \item Expand the function 
    $$f(z)= \frac{1}{z+1}$$ as a Taylor series about the point $z_0 = 0$ 
    (i.e. find the Maclaurin expansion.). Find the region of convergence.\\

    \item Expand the same function of the previous exercise as a Taylor series about $z_0 = 1$ 
    and find the region of convergence.

    \item Expand the function
    $$f(z) = \frac{z + 1}{z^2 - z - 6}$$
    as a Taylor series to order $(z - 2)^3$ about the point $z = 2$. Find the region of convergence. (Hint: you can rewrite the fiven function as a sum of two terms, using partial fractions). 

  \end{enumerate}

\end{document}
