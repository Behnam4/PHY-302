\documentclass[fleqn]{article}
\oddsidemargin 0.0in
\textwidth 6.0in
\thispagestyle{empty}
\usepackage{import}
\usepackage{amsmath}
\usepackage{graphicx}
\usepackage{flexisym}
\usepackage{amssymb}
\usepackage{bigints} 
\usepackage[english]{babel}
\usepackage[utf8x]{inputenc}
\usepackage{float}
\usepackage[colorinlistoftodos]{todonotes}

\definecolor{hwColor}{HTML}{AD53BA}

\begin{document}

  \begin{titlepage}

    \newcommand{\HRule}{\rule{\linewidth}{0.5mm}}

    \center



    \textsc{\LARGE Arizona State University}\\[1.5cm]

    \textsc{\LARGE Mathematical Methods For Physics II }\\[1.5cm]


    \begin{figure}
      \includegraphics[width=\linewidth]{asu.png}
    \end{figure}


    \HRule \\[0.4cm]
    { \huge \bfseries Homework Twelve}\\[0.4cm] 
    \HRule \\[1.5cm]

    \textbf{Behnam Amiri}

    \bigbreak

    \textbf{Prof: Cecilia Lunardini}

    \bigbreak


    \textbf{{\large \today}\\[2cm]}

    \vfill

  \end{titlepage}

  \textbf{Part A}
  \begin{enumerate}

    \item Find the poles in the following functions and determine their orders:
      \begin{enumerate}
        \item  $\frac{1}{z^2 - 2z + 1}$ , 
        \item  $\frac{1}{z^2 - 2z - 1}$ , 
        \item   $\frac{e^{(z-1)} - 1}{(z^2 + 1) (z - 1)}$  (Hint: be careful when analyzing the behavior at $z=1$, does the function diverge in that point? )
      \end{enumerate}

    \item Show that, if $u$ and $v$ are the real and imaginary parts of the analytic function, f$ (z)$,
    the family of curves, $u(x, y) = $constant, is orthogonal to the family of curves, $v(x, y) =$ constant.\\
    (hint: Use the Cauchy-Riemann conditions to show that the slope of the curve $u = c$ is the negative inverse of the slope of the curve $v = d$ (with $d$ and $c$ being constants) at the same point $(x, y)$). 


    \item {\bf Bonus (10 points credit): } reconsider the example on branch cuts shown in class (See class 23 Module and page 836 of the textbook) 
    and discuss the two possible choices of branch cuts, shown in fig. 24.2 (b) and 24.2 (c) of the book. Show that, 
    for appropriate (different) choices of the intervals for the phases of $z-i$ and $z+i$ (e.g., $\theta_1 \in [ -\pi/2 , 3 \pi/2]$), 
    the points along the branch cuts correspond to the points where the function is not single valued (i.e., the limit of the 
    function from one side of the cut is different from the limit of the function from the other side of the cut).  
    Computer graphics or high quality, creative by-hand graphics is strongly encouraged. 

  \end{enumerate}

  \pagebreak

  \textbf{Part B}
  \begin{enumerate}
    \item Soon
  \end{enumerate}

\end{document}
